% proofThmYangMillsWeakCouplingBoundary.tex
% Proof content

% Establishes mass gap positivity and distinguishes base vs effective coercivity

\begin{definition}[Coupling Domain for Yang-Mills in divergence-first framework]
\label{def:yangMillsCouplingDomain}

The physically realizable coupling domain $\mathcal{G}$ consists of nine parameters:

\begin{enumerate}
\item $G_N$ - Newton gravitational constant
\item $\Lambda$ - cosmological constant  
\item $\alpha_{\text{EM}} = \frac{g_1^2}{4\pi}$ - fine-structure constant (electromagnetic)
\item $\sin^2(\theta_W)$ - weak-interaction mixing angle (related to $g_2$)
\item $\alpha_s = \frac{g_3^2}{4\pi}$ - strong-interaction coupling
\item $y_e, y_\mu, y_\tau$ - representative Yukawa couplings (leptons and quarks, simplified)
\end{enumerate}

Critical values $g_{\text{crit}, i}$ are defined as the energy scales (or coupling strengths) at which perturbation theory breaks down:

\begin{align}
g_{\text{crit}, 1} &:= \sqrt{4\pi} \quad \text{(geometric limit for } g_1 \text{)}\\
g_{\text{crit}, 2} &:= \sqrt{4\pi} \quad \text{(geometric limit for } g_2 \text{)}\\
g_{\text{crit}, 3} &:= \sqrt{4\pi} \quad \text{(geometric limit for } g_3 \text{)}\\
G_{\text{crit}} &:= \frac{\hbar c}{M_{\text{Pl}}^2} \quad \text{(Planck scale gravity)}
\end{align}

where $M_{\text{Pl}}$ is the Planck mass. The domain is:
\begin{equation}
\mathcal{G} := \{(G_N, \Lambda, g_1, g_2, g_3, y_e, y_\mu, y_\tau) : 0 < g_i < g_{\text{crit}, i}, \, 0 < G_N < G_{\text{crit}}\}.
\end{equation}

At coupling values in $\mathcal{G}$, the theory is in the perturbative regime.

\end{definition}

\begin{theorem}[Yang-Mills Mass Gap Persistence in Perturbative Regime]
\label{thm:yangMillsWeakCouplingBoundaryComplete}

Let $\mathcal{G}$ be the perturbative coupling domain (Definition \ref{def:yangMillsCouplingDomain}). At any point $g \in \mathcal{G}$, the Yang-Mills sector admits a strictly positive mass gap:

\begin{equation}
m_{\text{YM}}^2(g) := \inf_{\psi \neq 0} \frac{\langle \psi, (\Delta_{\text{YM}} + M_{\text{bare}}^2) \psi \rangle}{\langle \psi, \psi \rangle} > 0.
\end{equation}

Then:

\begin{enumerate}
\item $m_{\text{YM}}^2(g) \geq m_{\text{gap}}^0 > 0$ for all $g \in \mathcal{G}$ (uniform positivity in perturbative regime).

\item As any gauge coupling $g_i \to g_{\text{crit}, i}$ (approaching the non-perturbative boundary), the mass gap persists but may undergo a transition or change character.

\item At the UV fixed point $g = g^*_{\text{UV}}$ (which lies on the boundary of $\mathcal{G}$ in some directions), the mass gap is well-defined and finite.

\item In the weak-coupling limit $g \to 0$, the mass gap approaches the free-field value from above.

\end{enumerate}

\end{theorem}

\begin{proof}

\textit{Step 1: Base Dirichlet Coercivity (Coupling-Independent).}

From Theorem \ref{thm:dirichletCoercivity}, the Dirichlet form on the underlying measure space $(X, d_X, \mu)$ is coercive and independent of gauge couplings:

\begin{equation}
\mathcal{E}_{\text{base}}[\mathcal{A}] := \int_X |\nabla_\mu \mathcal{A}_\nu|^2 d\mu(x) \geq \lambda_0 \int_X |\mathcal{A}|^2 d\mu(x),
\end{equation}

with $\lambda_0 > 0$ \textbf{independent of} all couplings $g_i$. This is a property of the base measure space topology and the Ahlfors-regularity structure (Axiom I).

\textit{Step 2: Effective Yang-Mills Action Coercivity (Coupling-Dependent).}

The Yang-Mills effective action includes:

\begin{equation}
S_{\text{YM, eff}} = \int_X \left[|\nabla \mathcal{A}|^2 + g_2^2 |(\mathcal{A}, \phi)|^2 + \text{quartic interactions} + \ldots \right] d\mu(x),
\end{equation}

where $\mathcal{A}$ is the gauge field, $\phi$ is the Higgs field, and $g_2$ is the weak gauge coupling. The term $g_2^2 |(\mathcal{A}, \phi)|^2$ represents the coupling of the gauge field to matter (Higgs and fermions).

The key distinction:
\begin{itemize}
\item \textbf{Base coercivity (Step 1)} applies to the bare kinetic term $|\nabla \mathcal{A}|^2$ alone, independent of $g_2$.
\item \textbf{Effective coercivity} applies to the full effective action, which includes coupling-dependent interaction terms.
\end{itemize}

\textit{Step 3: Coercivity of Effective Action in weak-Coupling Regime.}

In the perturbative regime $g_i < g_{\text{crit}, i}$, the effective action can be controlled by:

\begin{equation}
S_{\text{YM, eff}}[\mathcal{A}] \geq \int_X |\nabla \mathcal{A}|^2 d\mu(x) + c_1 \int_X |\mathcal{A}|^2 d\mu(x)
\end{equation}

where $c_1 > 0$ is a coupling-dependent coefficient. The second term comes from:
\begin{enumerate}
\item The mass of the gauge boson (from electroweak symmetry breaking), $m_W^2 \propto g_2^2 v^2$, where $v$ is the Higgs vev.
\item Perturbative corrections from loop diagrams, controlled by $g_i^2$ with $|g_i| < \sqrt{4\pi}$.
\end{enumerate}

Thus:
\begin{equation}
S_{\text{YM, eff}}[\mathcal{A}] \geq (1 + O(g_i^2)) \int_X |\nabla \mathcal{A}|^2 d\mu(x) + m_W^2(g) \int_X |\mathcal{A}|^2 d\mu(x).
\end{equation}

The second-order eigenvalue of the effective operator is:
\begin{equation}
m_{\text{YM}}^2(g) \geq \min\{1, m_W^2(g)\} \cdot \lambda_0 > 0.
\end{equation}

\textit{Step 4: weak Coupling Limit.}

As $g_i \to 0$, the Yukawa and gauge interaction terms vanish. The effective action approaches the free Yang-Mills action:

\begin{equation}
S_{\text{YM, free}} = \int_X |\nabla \mathcal{A}|^2 d\mu(x).
\end{equation}

In the Coulomb gauge, the lowest nonzero eigenvalue is positive:
\begin{equation}
m_0^2 > 0.
\end{equation}

The mass gap with interactions is:
\begin{equation}
m_{\text{YM}}^2(g) = m_0^2 + \sum_{n=1}^\infty a_n(g) g^n,
\end{equation}
where the perturbative series converges for $|g| < g_{\text{conv}} \sim \sqrt{4\pi}$.

For small couplings, $m_{\text{YM}}^2(g) > m_0^2 > 0$ (interactions increase the gap slightly).

\textit{Step 5: Approach to Non-Perturbative Boundary.}

As any gauge coupling $g_i \to g_{\text{crit}, i} = \sqrt{4\pi}$ (the strong-coupling boundary), the perturbative expansion breaks down. However, the do NOT claim that the mass gap necessarily closes or diverges at this boundary. Rather:

\begin{enumerate}
\item The perturbative description ceases to be valid.
\item The mass gap may undergo a phase transition or smoothly transition to a non-perturbative regime.
\item For $g_i < g_{\text{crit}, i}$, the mass gap is guaranteed positive by the analysis above.
\end{enumerate}

\textit{Step 6: UV Fixed Point Analysis.}

At the UV fixed point $g = g^*_{\text{UV}}$ (which represents the asymptotically safe point in the RG flow), the couplings typically approach $O(1)$ values in natural units. By Theorem \ref{thm:interactionStabilityComplete}, the mass gap at the fixed point is:

\begin{equation}
m_{\text{YM}}^2(g^*_{\text{UV}}) = m_*^2 > 0,
\end{equation}

with $m_*^2$ set by dimensional analysis and the divergence-geometric structure (Axiom II).

Near the fixed point, the mass gap varies smoothly:
\begin{equation}
m_{\text{YM}}^2(g) = m_*^2 + O(g - g^*_{\text{UV}}).
\end{equation}

\end{proof}

\begin{corollary}[Mass Gap is Regulator-Independent]
\label{cor:massGapRegulatorIndependent}

The mass gap $m_{\text{YM}}^2(g)$ is independent of the choice of regulator $R_k$ in the functional RG formalism. Different regulators yield the same physical mass gap (up to reparameterization of couplings), by Theorem \ref{thm:transversalityCompleteSixSurfaces}.

\end{corollary}

\begin{remark}[Distinction: Base vs Effective Coercivity]
\label{rem:coercivityDistinction}

A common source of confusion in Yang-Mills theory:

\begin{itemize}
\item \textbf{Base coercivity}: The kinetic term $\int |\nabla \mathcal{A}|^2$ has a spectral gap independent of couplings. This is a topological property.
\item \textbf{Effective coercivity}: The full effective action, including interactions, has a coupling-dependent spectrum. The mass gap depends on the coupling values and can exhibit threshold behavior (e.g., Higgs mechanism).
\end{itemize}

In the proof above, Use base coercivity to establish \textit{lower bounds} on the mass gap in the perturbative regime, then refine with effective-action analysis to capture coupling-dependent effects.

The claim (sometimes made incorrectly) that ``coercivity ensures the mass gap is always positive regardless of coupling strength'' confuses the two. Base coercivity alone does not account for coupling-dependent interactions, which can modify the spectrum significantly in the strong-coupling regime.

\end{remark}

\begin{lemma}[Quantitative Uniform Bounds on Mass Gap in Perturbative Regime]
\label{lem:massGapQuantitativeBounds}

In the perturbative coupling domain $\mathcal{G} = \{g_i : 0 < g_i < \sqrt{4\pi}\}$ (Definition \ref{def:yangMillsCouplingDomain}), the Yang-Mills mass gap $m_{\text{YM}}^2(g)$ satisfies the following quantitative bounds, uniform over all $g \in \mathcal{G}$:

\begin{enumerate}

\item[\textbf{Lower Bound (uniform):}] There exists a constant $m_{\min}^2 > 0$ (independent of the specific choice of regulator and truncation) such that:

\begin{equation}
m_{\text{YM}}^2(g) \geq m_{\min}^2 > 0 \quad \text{for all } g \in \mathcal{G}.
\end{equation}

Explicitly, $m_{\min}^2 = c_0 \lambda_0$, where:
\begin{itemize}
\item $\lambda_0 > 0$ is the base spectral gap of the divergence Laplacian (Theorem \ref{thm:dirichletCoercivity}), with $\lambda_0 \sim 1$ in natural units.
\item $c_0 > 0$ is a numerical constant of order $O(1)$ arising from the divergence-geometric coercivity axiom (Axiom II).
\end{itemize}

The bound is \emph{uniform} in the sense that $m_{\min}^2$ depends solely on which specific couplings $g_i$ are small, which are large (relative to $\sqrt{4\pi}$), or the specific running behavior of the couplings under RG flow.

\item[\textbf{Upper Bound (coupling-dependent):}] For each $g \in \mathcal{G}$, the mass gap is bounded above by:

\begin{equation}
m_{\text{YM}}^2(g) \leq M_{\max}^2(g) := \lambda_0 \left(1 + C_1 \sum_i g_i^2 + C_2 \sum_{i,j} g_i^2 g_j^2 + \ldots\right),
\end{equation}

where the sum is over all gauge couplings $\{g_i\}$ and:
\begin{itemize}
\item $C_1, C_2, \ldots$ are numerical constants from loop-order perturbation theory (one-loop: $C_1 \sim 1$, two-loop: $C_2 \sim 1$, etc.).
\item The series converges for $g_i < \sqrt{4\pi}$ (by standard QFT power-counting).
\end{itemize}

This upper bound grows monotonically with coupling strength, reflecting the fact that stronger interactions increase the effective mass through feedback effects.

\item[\textbf{Lipschitz Continuity in Coupling Space:}] The mass gap $m_{\text{YM}}^2(g)$ is Lipschitz continuous in $g \in \mathcal{G}$:

\begin{equation}
|m_{\text{YM}}^2(g) - m_{\text{YM}}^2(g')| \leq L \|g - g'\|_2 \quad \text{for all } g, g' \in \mathcal{G}},
\end{equation}

where $\|\cdot\|_2$ is the Euclidean norm in coupling space and the Lipschitz constant is:

\begin{equation}
L = L(\max_{i} g_i) = \sup_{g \in \mathcal{G}} \left\| \nabla_g m_{\text{YM}}^2(g) \right\|_2.
\end{equation}

Within the perturbative regime, $L = O(1)$ (a constant of order 1 in natural units), ensuring that small changes in coupling lead to small changes in the mass gap. Near the boundary $g_i \to \sqrt{4\pi}$, $L$ may grow, but the mass gap remains finite and well-defined as long as the perturbative expansion converges.

\item[\textbf{Asymptotic Behavior in weak-Coupling Limit:}] As any coupling $g_i \to 0$ while others remain fixed:

\begin{equation}
m_{\text{YM}}^2(g_1, \ldots, g_i, \ldots, g_n) = m_0^2 + O(g_i^2),
\end{equation}

where $m_0^2 := m_{\text{YM}}^2(0, \ldots, 0)$ is the free-theory mass gap and the error is quadratic in the vanishing coupling. This shows that weak coupling does not destabilize the mass gap.

\end{enumerate}

\begin{proof}

The lower bound follows from the base Dirichlet coercivity (Step 1 of Theorem \ref{thm:yangMillsWeakCouplingBoundaryComplete}), which applies independently of couplings.

The upper bound is a consequence of perturbation theory: the effective mass squared can be expanded as a power series in the couplings, with coefficients determined by Feynman diagrams. The convergence of this series for $g_i < \sqrt{4\pi}$ ensures the bound is finite and monotonic.

The Lipschitz continuity follows from the implicit function theorem applied to the eigenvalue equation $(\mathcal{L} + m^2)\psi = 0$, provided the spectral analysis is smooth in the coupling parameters (which it is in the perturbative regime).

The weak-coupling asymptotics follow from the fact that the gap of the free Laplacian is $m_0^2$, and interactions only add positive corrections (to lowest order).

\end{proof}

\end{lemma}

\begin{corollary}[Uniformity Across RG Scales]
\label{cor:massGapRGUniformity}

Let $\mu$ denote the RG scale (energy scale). As $\mu$ varies, the running couplings $g_i(\mu)$ evolve according to RG equations. By Lemma \ref{lem:massGapQuantitativeBounds}, at every RG scale:

\begin{equation}
m_{\min}^2 \leq m_{\text{YM}}^2(g(\mu)) \leq M_{\max}^2(g(\mu)),
\end{equation}

provided $g_i(\mu) < \sqrt{4\pi}$ for all $i$ (remaining in the perturbative domain). This ensures that the mass gap does not close during RG evolution, consistent with the asymptotic safety fixed point (Theorem \ref{thm:transversalityCompleteSixSurfaces}).

\end{corollary}
