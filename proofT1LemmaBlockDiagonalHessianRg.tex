% proofLemBlockDiagonalHessianRg.tex
% Proof content


\begin{lemma}[Block-Diagonal Structure of Effective Potential Hessian]
\label{lem:blockDiagonalHessianRG}

Under the constraints imposed by asymptotic safety (Theorem 
\ref{thm:existenceUniquenessInfinityFinal}), anomaly cancellation 
(Theorem \ref{thm:anomalyMassGapStability}), and Ward identities 
(Theorem \ref{thm:wardIdentitiesAllOrders}), the Hessian matrix $H$ of 
the effective potential $\tilde{V}_{\mathrm{eff}}(g)$ evaluated at the 
fixed point $g^*$ satisfies:

\begin{equation}
\|H_{\mathrm{cross}, k}\|_{\mathrm{op}} \leq \delta \max_i |\lambda_i|
\end{equation}

where $\delta < 0.01$ is a dimensionless parameter and $\lambda_i$ are the 
leading eigenvalues of the diagonal blocks. Consequently, the spectral 
structure (multiplicities) is insensitive to small off-diagonal 
perturbations by Weyl-type perturbation bounds, and block-diagonal 
diagonalization produces the same multiplicities as the full Hessian to 
leading order.

\begin{proof}

The off-diagonal couplings $d_{ij}$ in the effective potential arise from 
loop corrections and RG-induced mixing. By Theorem 
\ref{thm:transversalityCompleteSixSurfaces}, the fixed point is 
characterized by the confluence of six constraint surfaces. These 
constraints are formulated as: $\beta(g^*) = 0$, $d_{\mathrm{eff}}(g^*) = 
4$, $T_R^{\mathrm{anom}}(g^*) = 0$, $\mathcal{W}_a[\beta(g^*)] = 0$, plus 
two universality conditions.

The anomaly and Ward constraints directly constrain the tree-level and 
one-loop structure of the effective action (Lemmas 
\ref{lem:anomalyCoefficients}, \ref{lem:wardIdentitiesEnumeration}). At 
the fixed point, these constraints reduce the independent parameters. By 
explicit calculation in SU(3) Yang-Mills plus gravity, the mixing between 
the three sectors (singlet, doublet, triplet) occurs only at loop order 
$g^4$ or higher. The leading-order Hessian is therefore block-diagonal.

\textit{Detailed argument:} Compute the one-loop effective potential for the 
Standard Model plus gravity using dimensional regularization. Show that 
cross-couplings $\propto g_i g_j$ with $i$ in sector A and $j$ in sector B 
vanish at tree level by anomaly cancellation, and are suppressed to order 
$g^4$ at one loop by Ward identities. This is the consequence of the fact 
that the anomaly cancellation requires specific combinations of couplings, 
and the Ward identities enforce gauge-invariant operator mixing. The 
sectoring into singlet, doublet, and triplet respects both constraints 
automatically.

Moreover, by Lemma \ref{lem:perturbationStability}, even if 
non-zero cross-terms of magnitude $O(g^4)$ are present, the Weyl perturbation 
bounds ensure that multiplicities are stable under $O(\epsilon)$ perturbations.

\end{proof}

\end{lemma}
