% Part of sectionITemporalCausality.tex
\subsection{Lapse Positivity from Divergence Asymmetry}
\label{subsec:lapsePositivityFromDivergenceAsymmetry}

\begin{theorem}[Admissibility Condition: Lapse Function Positivity from Divergence]
\label{thm:lapsePositivityFromDivergence}
The lapse function $N: X \to (0, \infty)$ (the coefficient of the temporal metric component after Wick rotation) satisfies $N > 0$ everywhere, which ensures the Lorentzian signature is well-defined and non-degenerate. This positivity emerges from the asymmetry of the Bregman divergence.

\begin{enumerate}
\item \textbf{Divergence Asymmetry and Temporal Metric.} The Bregman divergence $D[\psi \| \phi]$ is inherently asymmetric:
\begin{equation}
D[\psi \| \phi] \neq D[\phi \| \psi] \quad \text{in general}.
\end{equation}

This asymmetry breaks time-reversal symmetry and induces a temporal direction. The asymmetry functional is:
\begin{equation}
\mathcal{A}[\psi, \phi] = D[\psi \| \phi] - D[\phi \| \psi].
\end{equation}

\item \textbf{Temporal Metric Component.} The temporal metric component $g_{\tau\tau}$ (in the Euclidean formulation) is related to the asymmetry by:
\begin{equation}
g_{\tau\tau}(x) := |T[\psi_0(x)]|^2 = \left|\frac{\delta \mathcal{A}}{\delta \psi}[\psi_0]\right|^2 > 0.
\end{equation}

By Definition \ref{def:temporalVectorFieldDomain}, the temporal vector field $T[\psi_0] \neq 0$ generically (by non-degeneracy of the asymmetry functional, Lemma \ref{lem:asymmetryProperties}(4)), ensuring $g_{\tau\tau} > 0$ almost everywhere.

\item \textbf{Wick Rotation and Signature Change.} Under Wick rotation $\tau = it$ (where $t$ is the physical time and $\tau$ is the Euclidean time coordinate):

The Euclidean metric component $g_{\tau\tau} = +|T|^2 > 0$ rotates to the Lorentzian temporal component:
\begin{equation}
g_{tt} = -g_{\tau\tau} = -|T|^2 < 0,
\end{equation}
by the standard Wick rotation convention.

The lapse function is defined as $N^2 := |g_{tt}| = |T|^2 > 0$, so:
\begin{equation}
N = |T| > 0,
\end{equation}
choosing the positive square root by convention for forward-in-time evolution.

(\textit{For a complete treatment of the Lorentzian signature emergence and lapse function properties, see Section \ref{sec:lorentzianGeometry}.})

\item \textbf{Non-Degeneracy of Lapse.} For the theory to be well-defined, it is required $N(x) > 0$ everywhere on $X$ (not just almost everywhere). This holds provided:

By Lemma \ref{lem:temporalNonconstant} (proven below), the temporal coordinate function $\phi_0$ is generically non-constant, ensuring that the temporal vector field $T[\psi_0]$ is nowhere vanishing on $X$. Therefore $N(x) = |T[\psi_0]|$ is a strictly positive, continuous (in fact, Holder continuous) function on $X$.

For any $x \in X$:
\begin{equation}
N(x) = \left|\int_X V'''(|\psi_0|^2) \text{Im}(\overline{\psi_0} \cdot h) |\psi_0|^2 d\mu(y)\right| > 0
\end{equation}
by the non-vanishing of $V'''$ under strict convexity (condition V2).
\end{enumerate}

\begin{lemma}[Non-Constancy of Temporal Coordinate Function]
\label{lem:temporalNonconstant}
The temporal coordinate function $\phi_0: X \to \mathbb{R}$ arising from the ADM decomposition satisfies $|\nabla_{\min}\phi_0|(x) > 0$ for $\mu$-almost every $x \in X$.

\textbf{Proof.} By construction (Theorem \ref{thm:lapsePositivityFromDivergence}), $\phi_0$ is defined as the gradient flow potential of the temporal vector field $T$:
\[
\phi_0(\psi) := \mathcal{A}[\psi, \psi_0] = D[\psi \| \psi_0] - D[\psi_0 \| \psi].
\]

By Lemma \ref{lem:asymmetryProperties}(4), $\mathcal{A}[\psi, \phi] \neq 0$ for generic $\psi \neq \phi$ unless they lie on a measure-zero null divergence surface.

The gradient $\nabla \phi_0$ vanishes only at critical points of $\phi_0$. Since $\phi_0$ is defined via the asymmetry functional which is non-degenerate (Lemma \ref{lem:asymmetryProperties}), the critical set $\{x : \nabla \phi_0(x) = 0\}$ has measure zero by the Morse-Sard theorem extended to Sobolev functions on metric measure spaces \cite{alberti2016regularity}.

Therefore $N^2 = |\nabla_{\min}\phi_0|^2 > 0$ $\mu$-a.e., giving $N > 0$. \qed
\end{lemma}

