% abstract.tex
% Abstract of The Barg Theory manuscript


\begin{abstract}

This framework derives the entire Standard Model coupled to general relativity from two minimal axioms rooted in asymmetric Bregman divergence, a measure from information geometry that quantifies how probability distributions differ. The framework establishes that divergence generates spacetime geometry and all fundamental interactions. Geometric structure, quantum mechanics, gauge interactions, and gravitational dynamics emerge as necessary mathematical consequences of divergence axiomatization, without any assumption about dimensionality, signature, gauge symmetry, particle content, or forces.

The first axiom specifies a Polish metric measure space satisfying Ahlfors regularity and Poincaré inequality, ensuring well-behaved functional analysis without constraining dimension. The second axiom specifies a configuration space of square-integrable complex functions governed by a strictly convex generating functional. These two axioms suffice to generate all subsequent physics.

From the generating functional's strict convexity emerges an asymmetric Bregman divergence. This divergence encodes three independent information channels derived from the Hessian of the functional. These three channels, designated soft, bulk, and stiff modes, induce representation-theoretic structure via dihedral group D three symmetry. The three-generation structure of the Standard Model receives unique determination. The requirement that all triangle anomalies cancel in quantum field theory uniquely determines the gauge group structure. This gauge group admits exactly and only three anomaly-free complete generations of quarks and leptons, rendering the number of generations equal to three as a mathematical necessity of anomaly cancellation. The three-fold structure encoded in the divergence's Hessian manifests mathematically consistent with the three-fold generation pattern through representation theory of the dihedral group acting on the three Hessian eigenvalue scales. The Floer homology of the configuration space decomposes into three topologically distinct chain complexes, each corresponding to one generation's vacuum sector, providing a topological expression of this same three-generation structure. Empirical data on Higgs vacuum stability, CP-violation, and precision electroweak measurements confirm consistency with the derived value of three generations.

Divergence asymmetry breaks time-reversal symmetry and induces causal ordering. Polarization of the divergence functional yields a quadratic form defining a Dirichlet form structure, which determines a unique self-adjoint spectral operator. Eigenfunctions of this operator satisfy regularity conditions that constrain spacetime dimensionality. Hölder continuity of eigenfunctions forces dimension strictly below four. Yang-Mills renormalizability demands dimension at most four. Chiral anomaly cancellation requires even dimensionality. Graviton propagation requires dimension at least four. These four independent constraints converge to uniquely determine four-dimensional spacetime. The Osterwalder-Schrader reconstruction theorem performs analytic continuation from Euclidean to Lorentzian signature, yielding one temporal and three spatial directions.

Riemannian metric geometry emerges automatically through the Carré du Champ formalism applied to eigenfunctions. This emergence follows necessarily from the dimensional constraint. Quantum mechanics arises through path integral formulation on the divergence-induced measure. Matter fields stabilize into localized solitonic configurations with discrete energy spectra. The one-loop quantum correction generates the Einstein-Hilbert action, establishing general relativity as a quantum effect.

Gauge interactions emerge from symmetry transformations preserving the divergence structure. The requirement of anomaly cancellation in quantum field theory uniquely determines the gauge group structure. This gauge group satisfies six anomaly cancellation conditions while maintaining consistency with the derived dimension and generation structure.

The Yang-Mills mass gap receives establishment through four logically independent mechanisms with mutual redundancy. The first mechanism uses asymptotic safety at the ultraviolet fixed point to establish weak coupling. The second mechanism applies functional renormalization group analysis via bifurcation of the effective average action, independent of asymptotic safety. The third mechanism applies direct spectral analysis on the Polish space via Dirac operator eigenvalue bounds, establishing the gap without renormalization group methods. The fourth mechanism uses geometric bounds on Ricci curvature in the emergent metric to prove confinement and the gap. These mechanisms provide multiple independent proofs, with mechanisms two and three each furnishing unconditional proofs of the gap regardless of asymptotic safety verification.

Ultraviolet finiteness follows from asymptotic safety without extra dimensions. The renormalization group flow exhibits a non-Gaussian fixed point determined by transversality of six constraint surfaces in coupling space: divergence rigidity, spectral dimension consistency, information-geometric monotonicity, anomaly cancellation, lattice-continuum limit matching, and Ward identity preservation. The dimensional flow under renormalization group evolution shows effective dimension flowing from fractal-like ultraviolet structure near two point seven to smooth infrared dimension three, providing independent verification of four-dimensional spacetime.

A self-adjoint operator construction emerges whose eigenvalues correspond to the zeros of the Riemann zeta function on the critical line. This operator derives from the spectral theory of the divergence-induced Laplacian. The existence of this operator satisfies the conditions sufficient for the Riemann Hypothesis.

Conventional approaches to quantum gravity and unification treat spacetime dimension, signature, gauge groups, and particle content as external inputs or empirical choices. This framework derives each as a unique mathematical necessity. The Standard Model coupled to general relativity emerges as the sole consistent realization of Axioms I and II through mathematical rigor and logical necessity.

\end{abstract}