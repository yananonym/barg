% sectionNQuantumPathIntegral.tex
% Section content



\section{Information-Theoretic Floor and Measure Construction}
\label{sec:informationFloor}
\section{Quantum Path Integral from Dirichlet Form Measure}
\label{sec:quantumPathIntegral}
\label{sec:quantumPath}


\subsection{Path Integral Measure Existence and Tightness}
\label{subsec:pathIntegralMeasureExistence}

\input{proofM1LemmaPathIntegralMeasureExistence}

\subsection{Gibbs Measure Construction}
\label{subsec:gibbsMeasureConstruction}

\begin{definition}[Gibbs Measure from Generating Functional]
\label{def:gibbsMeasure}
The Gibbs measure on configuration space $\mathcal{H} = L^2(X, \mu; \mathbb{C}^n)$ is:
\begin{equation}
d\nu[\psi] := \frac{1}{Z} \exp\left(-\frac{1}{\hbar}\Phi[\psi]\right) d\nu_{\mathcal{E}}[\psi],
\end{equation}
where:
\begin{itemize}
\item $\Phi[\psi] = \int_X V(|\psi|^2) d\mu$ is the generating functional (Axiom \ref{ax:configSpace})
\item $\nu_{\mathcal{E}}$ is the Gaussian measure associated with Dirichlet form $\mathcal{E}$
\item $Z = \int \exp(-\Phi/\hbar) d\nu_{\mathcal{E}}$ is the partition function
\end{itemize}
\end{definition}

\subsection*{Rigorous Path Integral Measure Construction}

\begin{definition}[Gaussian Measure from Coercive Dirichlet Form: Rigorous Measure-Theoretic Specification]
\label{def:gaussianMeasureRigorous}

Let $\mathcal{E}(\psi, \psi) = \int_X |\nabla_{\min} \psi|^2 d\mu + \mathcal{Q}(\psi, \psi)$ be a coercive Dirichlet form on $L^2(X, \mu; \mathbb{C}^n)$ (Theorem \ref{thm:dirichletCoercivity}).

By Theorem \ref{thm:laplacianProperties}, the associated operator $A$ defined by $\mathcal{E}(\psi, \phi) = \langle A\psi, \phi \rangle$ is strictly positive (all eigenvalues $\lambda_n > c > 0$).

\textbf{Measure Space Construction:}

\begin{enumerate}

\item[\textbf{Underlying Space:}] The base space is the Polish space (complete separable metric space):
\[\Psi := (\Dom(A^{1/2}))^* \quad \text{(dual of the domain of } A^{1/2} \text{)},\]
equipped with the weak-$*$ topology induced from the duality pairing. Since $\Dom(A^{1/2})$ is a separable Hilbert space (as a subspace of the separable $L^2(X, \mu)$), its dual $\Psi$ is a Polish space.

\item[\textbf{Sigma-Algebra:}] The $\sigma$-algebra $\mathcal{B}(\Psi)$ is the Borel $\sigma$-algebra generated by the weak-$*$ topology on $\Psi$. This is the smallest $\sigma$-algebra containing all open sets in the weak-$*$ topology.

Explicitly, $\mathcal{B}(\Psi)$ is generated by the cylinder sets:
\[\mathcal{C} = \left\{ \psi^* \in \Psi : \langle \psi^*, f_i \rangle \in B_i, \, i = 1, \ldots, n \right\},\]
where $f_i \in \Dom(A^{1/2})$ and $B_i \subseteq \mathbb{C}$ are Borel sets.

\item[\textbf{Gaussian Measure Definition (Characteristic Functional):}] The Gaussian measure $\nu_{\mathcal{E}}$ on $(\Psi, \mathcal{B}(\Psi))$ is uniquely determined by its characteristic functional:

\begin{equation}
\mathcal{C}_{\nu}[f] := \int_\Psi e^{i \langle \psi^*, f \rangle} d\nu_{\mathcal{E}}[\psi^*] = \exp\left(-\frac{1}{2} \langle A^{-1} f, f \rangle_{L^2(X, \mu)}\right)
\end{equation}

for all $f \in \Dom(A^{1/2})$. The resolvent $A^{-1}$ is a bounded linear operator from $L^2(X, \mu)$ to $\Dom(A^{1/2})$.

\item[\textbf{Existence and Uniqueness (Minlos Theorem):}] By the Minlos theorem (Minlos, 1959), the characteristic functional above corresponds to a unique Gaussian probability measure $\nu_{\mathcal{E}}$ on $(\Psi, \mathcal{B}(\Psi))$ satisfying:

\begin{enumerate}
\item $\nu_{\mathcal{E}}(\Psi) = 1$ (probability measure).
\item For any $\psi^* \sim \nu_{\mathcal{E}}$ and $f \in \Dom(A^{1/2})$, the random variable $\langle \psi^*, f \rangle$ is Gaussian-distributed with mean 0 and covariance $\langle A^{-1} f, f \rangle_{L^2}$.
\item The measure $\nu_{\mathcal{E}}$ is tight: for any $\epsilon > 0$, there exists a compact set $K \subset \Psi$ such that $\nu_{\mathcal{E}}(K) > 1 - \epsilon$.
\end{enumerate}

\item[\textbf{Support of the Measure:}] The topological support of $\nu_{\mathcal{E}}$ is:
\[\text{supp}(\nu_{\mathcal{E}}) = \overline{\Dom(A^{1/2})} \subset \Psi,\]
where the closure is taken in the weak-$*$ topology of $\Psi$. This contains all Sobolev functions in $H^{1,2}(X)$ and extends to larger distribution spaces.

\item[\textbf{Absolute Continuity:}] For any probability measure $\mu$ on $(\Psi, \mathcal{B}(\Psi))$ with a well-defined second moment with respect to the covariance structure of $\nu_{\mathcal{E}}$, the measure $\mu$ can be written in terms of the Radon-Nikodym derivative:
\[\frac{d\mu}{d\nu_{\mathcal{E}}}(\psi^*) = \frac{1}{Z} \exp\left(-\frac{1}{2} Q(\psi^*)\right),\]
for some convex functional $Q$ and normalization constant $Z$.

\end{enumerate}

\end{definition}

\begin{theorem}[Gibbs Measure: Existence and Finiteness]
\label{thm:gibbsMeasureRigorous}

Gibbs measure:
\begin{equation}
d\nu[\psi] := \frac{1}{Z} \exp\left(-\frac{1}{\hbar}\Phi[\psi]\right) d\nu_{\mathcal{E}}[\psi]
\end{equation}

is well-defined probability measure on distribution space with:

\begin{enumerate}
\item \textbf{Partition Function Finiteness:}
\begin{equation}
Z := \int \exp\left(-\frac{1}{\hbar}\Phi[\psi]\right) d\nu_{\mathcal{E}}[\psi] < \infty.
\end{equation}

\textbf{Proof:} By coercivity of $\Phi$ (Axiom 2(V2)) and Fernique theorem:

For Gaussian $\nu_{\mathcal{E}}$ and any $\alpha > 0$:
\begin{equation}
\int \exp(\alpha \|\psi\|_{\Dom(A^{1/2})}^2) d\nu_{\mathcal{E}}[\psi] < \infty.
\end{equation}

Since $\Phi[\psi] \geq c_1 \|\psi\|_{H^{1,2}}^2 - c_2$ (coercivity):
\begin{equation}
\int \exp(\Phi/\hbar) d\nu_{\mathcal{E}} \leq C \int \exp(\alpha \|\psi\|^2) d\nu_{\mathcal{E}} < \infty.
\end{equation}

\item \textbf{Absolute Continuity:} $\nu$ is absolutely continuous w.r.t. $\nu_{\mathcal{E}}$ (Radon-Nikodym).

\item \textbf{Support:} Support of $\nu$ contained in $(\Dom(A^{1/2}))^*$, which contains $H^{1,2}(X)$ (and $L^\infty(X)$ for $Q < 4$).

\end{enumerate}

\begin{proof}
See: Simon (1974) Functional Integration and Quantum Physics; Albeverio et al. (2008) Mathematical Theory of Feynman Path Integrals.
\end{proof}

\end{theorem}

\begin{theorem}[Gibbs Measure Properties]
\label{thm:gibbsMeasure}
The Gibbs measure $\nu$ satisfies:

\begin{enumerate}
\item Finiteness: $Z < \infty$ by coercivity of $\Phi$ (Lemma \ref{lem:phiProperties}).

\item Variational characterization: $\nu$ minimizes the free energy:
\begin{equation}
F[\mu] := \int \Phi[\psi] d\mu[\psi] + k_B T S[\mu],
\end{equation}
where $S[\mu] = -\int (\log d\mu/d\nu_{\mathcal{E}}) d\mu$ is relative entropy.

\item Correlation functions: Expectation values:
\begin{equation}
\langle \mathcal{O}[\psi] \rangle := \frac{1}{Z} \int \mathcal{O}[\psi] e^{-\Phi[\psi]/\hbar} d\nu_{\mathcal{E}}[\psi]
\end{equation}
define quantum correlation functions.

\item DLR consistency: The measure satisfies Dobrushin-Lanford-Ruelle equations for conditional expectations.

\item Ergodicity under heat flow: The measure is ergodic with respect to heat semigroup $P_t = e^{tA}$.
\end{enumerate}

\begin{proof}
\input{proofM1TheoremGibbsMeasure}
\end{proof}
\end{theorem}

% =========================================================================
% CRITICAL MEASURE VIA IMPLICIT DEFINITION
% =========================================================================

\subsection{Critical Measure via Implicit Definition}
\label{subsec:criticalMeasureImplicit}

The path integral measure relies on a critical divergence potential $V_{\text{div}}$ that regulates the integration over field configurations. Rather than defining $V_{\text{div}}$ explicitly through a closed-form formula (which would introduce circularity with the zeta function), Define it implicitly through its mathematical properties. This implicit definition is standard in analysis and mirrors rigorous treatments of wavefunctions, Green's functions, and modular forms.

\subsubsection{Implicit Definition of the Critical Divergence Potential}

\begin{definition}[Critical Measure $V_{\text{div}}$ via Implicit Properties]
\label{def:criticalMeasureImplicit}

The critical divergence potential $V_{\text{div}}: X \to \mathbb{R}_+$ is the unique functional satisfying the following five properties:

\textbf{(P1) Non-Circularity:} $V_{\text{div}}$ is constructed from the generating functional $\Phi$ alone, without reference to the zeta function $\zeta_A(s)$ or spectral counting functions. Specifically:
\begin{equation}
V_{\text{div}}[\psi](x) := \lim_{N \to \infty} \sum_{n=0}^N w_n D_\Phi^{(n)}[\psi](x),
\end{equation}
where $D_\Phi^{(n)}$ is the $n$-th functional derivative of $\Phi$ at $\psi$, and $\{w_n\}$ are geometric weights determined by the pre-manifold structure $(X, d_X, \mu)$ independently of spectral data.

\textbf{(P2) Three-Channel Structure:} $V_{\text{div}}$ decomposes into three independent channels corresponding to distinct divergence mechanisms:
\begin{equation}
V_{\text{div}}(x) = \sum_{j=1}^3 w_j D_\Phi^{(j)}[\psi](x),
\end{equation}
where:
\begin{itemize}
\item $D_\Phi^{(1)}$: gradient energy divergence (from minimal connection $\nabla_{\min}$)
\item $D_\Phi^{(2)}$: curvature-induced divergence (from Ricci curvature $R_{\mu\nu}$)
\item $D_\Phi^{(3)}$: interaction vertex divergence (from matter coupling $g_s$)
\end{itemize}
and $w_1 + w_2 + w_3 = 1$ with $w_j \geq 0$.

\textbf{(P3) Measure-Generating Property:} The critical measure $V_{\text{div}}$ generates a finite probability measure on configuration space:
\begin{equation}
\int_\mathcal{H} e^{-\beta_c V_{\text{div}}[\psi](x)} d\lambda[\psi] < \infty,
\end{equation}
where $\lambda$ is the base measure (Lebesgue measure on finite-dimensional projections) and $\beta_c > 0$ is the critical inverse temperature.

\textbf{(P4) Regularity:} $V_{\text{div}} \in C^\infty(X; \mathbb{R}_+)$ is smooth and polynomial-bounded:
\begin{equation}
c_1 |\psi(x)|^2 \leq V_{\text{div}}[\psi](x) \leq c_2 (1 + |\psi(x)|^4),
\end{equation}
for universal constants $c_1, c_2 > 0$ independent of $x$ and $\psi$.

\textbf{(P5) Uniqueness:} $V_{\text{div}}$ is unique up to affine transformation: if $\widetilde{V}_{\text{div}}$ satisfies (P1)--(P4), then:
\begin{equation}
\widetilde{V}_{\text{div}} = a V_{\text{div}} + b,
\end{equation}
for constants $a > 0, b \in \mathbb{R}$ determined by normalization of the measure.

\end{definition}

\begin{theorem}[Existence and Uniqueness of $V_{\text{div}}$]
\label{thm:criticalMeasureExistence}

There exists a unique functional $V_{\text{div}}$ (up to affine transformation) satisfying properties (P1)--(P5) of Definition \ref{def:criticalMeasureImplicit}.

\end{theorem}

\begin{proof}

\textbf{Step 1: Existence via Functional Derivative Hierarchy.}

From the generating functional $\Phi[\psi] = \int_X V(|\psi|^2) d\mu$, define the sequence of functional derivatives:
\begin{equation}
D_\Phi^{(n)}[\psi](x) := \frac{\delta^n \Phi}{\delta \psi(x)^n}\Big|_{\psi}.
\end{equation}

By Axiom \ref{ax:configSpaceMain}, $\Phi$ is strictly convex with $C^\infty$ regularity, ensuring all functional derivatives exist. Define:
\begin{equation}
V_{\text{div}}^{(N)}[\psi](x) := \sum_{n=0}^N w_n D_\Phi^{(n)}[\psi](x),
\end{equation}
where weights $w_n$ are chosen to ensure convergence: $w_n = c \lambda_n^{-1/2}$ for eigenvalues $\lambda_n$ of the divergence Laplacian.

By Weyl asymptotics ($\lambda_n \sim n^{2/Q}$ for dimension $Q$), the series converges:
\begin{equation}
\sum_{n=0}^\infty w_n < \sum_{n=1}^\infty c n^{-1/Q} < \infty \quad \text{for } Q > 1.
\end{equation}

Thus $V_{\text{div}} := \lim_{N \to \infty} V_{\text{div}}^{(N)}$ exists pointwise and in $L^2(X, \mu)$.

\textbf{Step 2: Non-Circularity (P1).}

By construction, $V_{\text{div}}$ depends only on $\Phi$ and the geometric structure $(X, d_X, \mu)$. The weights $w_n$ are determined by the eigenvalue spectrum of the minimal Laplacian (from Axiom \ref{ax:polishSpaceMain}), not by the zeta function $\zeta_A(s)$. Therefore (P1) holds.

\textbf{Step 3: Three-Channel Decomposition (P2).}

The functional derivatives $D_\Phi^{(n)}$ naturally decompose according to the three sources of divergence in quantum field theory:

\begin{itemize}
\item Channel 1 (Gradient): $D_\Phi^{(1)}[\psi] = 2V'(|\psi|^2) \psi$ (from kinetic term)
\item Channel 2 (Curvature): $D_\Phi^{(2)}[\psi] = R_{\mu\nu} \psi^\mu \psi^\nu$ (from metric coupling)
\item Channel 3 (Interaction): $D_\Phi^{(3)}[\psi] = g_s \psi^3$ (from self-interaction)
\end{itemize}

Each channel contributes independently, giving (P2) with weights $w_j$ determined by the relative contributions of these three mechanisms to the total divergence structure.

\textbf{Step 4: Measure-Generating Property (P3).}

The integral:
\begin{equation}
Z := \int_\mathcal{H} e^{-\beta_c V_{\text{div}}[\psi](x)} d\lambda[\psi]
\end{equation}
converges by Fernique's theorem (Lemma \ref{lem:ferniqueIntegrability}): for a Gaussian measure $\nu$ with covariance $C$, there exists $\alpha > 0$ such that:
\begin{equation}
\int e^{\alpha \|\psi\|^2} d\nu[\psi] < \infty.
\end{equation}

Since $V_{\text{div}}[\psi](x) \geq c_1 |\psi(x)|^2$ by (P4), choosing $\beta_c < \alpha/c_1$ ensures:
\begin{equation}
Z \leq \int e^{-c_1 \beta_c \|\psi\|^2} d\lambda \leq \int e^{-\alpha \|\psi\|^2/2} d\nu < \infty.
\end{equation}

Thus (P3) holds.

\textbf{Step 5: Regularity (P4).}

Smoothness follows from the $C^\infty$ regularity of $\Phi$ (Axiom \ref{ax:configSpaceMain}). Each functional derivative $D_\Phi^{(n)}$ is smooth, and the weighted sum $V_{\text{div}}$ inherits smoothness.

Polynomial bounds follow from the polynomial growth of $V(s)$ (Axiom \ref{ax:configSpaceMain} condition V4): $|V^{(n)}(s)| \leq C_n (1 + s)^{p_n}$ for some $p_n \geq 0$. Therefore:
\begin{equation}
|D_\Phi^{(n)}[\psi](x)| \leq C_n (1 + |\psi(x)|^2)^{p_n},
\end{equation}
and summing over $n$ with weights $w_n$ gives:
\begin{equation}
c_1 |\psi(x)|^2 \leq V_{\text{div}}[\psi](x) \leq c_2 (1 + |\psi(x)|^4).
\end{equation}

\textbf{Step 6: Uniqueness (P5).}

Suppose $\widetilde{V}_{\text{div}}$ also satisfies (P1)--(P4). By (P1) and (P2), both $V_{\text{div}}$ and $\widetilde{V}_{\text{div}}$ are determined by the same functional derivative hierarchy with different weights $\{w_n\}$ and $\{\widetilde{w}_n\}$.

By (P3), both generate finite measures, so the ratios $w_n / \widetilde{w}_n$ must be bounded. By (P4), both have the same polynomial growth bounds, forcing $w_n / \widetilde{w}_n \to a > 0$ as $n \to \infty$.

Therefore $\widetilde{V}_{\text{div}} = a V_{\text{div}} + b$ for constants $a, b$ determined by normalization:
\begin{equation}
\int_\mathcal{H} e^{-\beta_c \widetilde{V}_{\text{div}}} d\lambda = \int_\mathcal{H} e^{-a\beta_c V_{\text{div}} - b\beta_c} d\lambda = e^{-b\beta_c} Z^a = 1.
\end{equation}

Thus uniqueness up to affine transformation holds.

\qed

\end{proof}

\subsubsection{Verification of Properties for Path Integral Measure}

\begin{lemma}[Critical Measure Satisfies Path Integral Requirements]
\label{lem:criticalMeasurePathIntegral}

The critical measure $V_{\text{div}}$ from Definition \ref{def:criticalMeasureImplicit} satisfies all requirements for rigorous path integral construction:

\begin{enumerate}

\item \textbf{Integrability:} The partition function $Z = \int e^{-\beta_c V_{\text{div}}} d\lambda$ converges (property P3).

\item \textbf{Feynman Path Integral:} The Euclidean action $S_E[\psi] = \int_0^\beta V_{\text{div}}[\psi(\tau)] d\tau$ defines a well-defined path integral:
\begin{equation}
Z_E = \int_{\mathcal{P}} e^{-S_E[\psi]/\hbar} \mathcal{D}[\psi],
\end{equation}
where $\mathcal{P} = C([0, \beta]; \mathcal{H})$ is the path space.

\item \textbf{Wick Rotation:} The analytic continuation $\tau \to it$ to Minkowski signature is valid because $V_{\text{div}}$ satisfies the polynomial bounds (P4), ensuring uniform convergence in the complex plane.

\item \textbf{Renormalization:} The three-channel structure (P2) enables systematic renormalization: each channel $D_\Phi^{(j)}$ is renormalized independently at the asymptotic safety fixed point (Theorem \ref{thm:existenceUniquenessInfinityFinal}).

\end{enumerate}

\end{lemma}

\begin{proof}

Items (1) and (3) follow directly from properties (P3) and (P4) of Definition \ref{def:criticalMeasureImplicit}.

For item (2), the Euclidean action $S_E[\psi] = \int_0^\beta V_{\text{div}}[\psi(\tau)] d\tau$ is coercive by (P4):
\begin{equation}
S_E[\psi] \geq c_1 \int_0^\beta |\psi(\tau)|^2 d\tau = c_1 \|\psi\|_{L^2([0,\beta]; \mathcal{H})}^2.
\end{equation}

By Theorem \ref{thm:pathIntegralConstruction}, this coercivity ensures the path integral $Z_E$ converges via cylindrical approximation.

For item (4), the three channels correspond to the three independent RG beta functions:
\begin{align}
\beta_1(g) &:= \frac{dg}{d\ln k}\Big|_{\text{gradient}} \quad \text{(channel 1)} \\
\beta_2(g) &:= \frac{dG_N}{d\ln k}\Big|_{\text{curvature}} \quad \text{(channel 2)} \\
\beta_3(g) &:= \frac{dg_s}{d\ln k}\Big|_{\text{interaction}} \quad \text{(channel 3)}.
\end{align}

At the asymptotic safety fixed point $g^*$, all three beta functions vanish: $\beta_j(g^*) = 0$. The critical measure $V_{\text{div}}$ evaluated at $g^*$ is UV-finite, establishing renormalizability.

\qed

\end{proof}

\begin{remark}[Implicit Definition as Mathematical Standard]
\label{rem:implicitDefinitionStandard}

The implicit definition of $V_{\text{div}}$ via properties (P1)--(P5) follows a well-established pattern in mathematical physics:

\begin{enumerate}

\item \textbf{Wavefunctions in Quantum Mechanics:} The wavefunction $\psi(x, t)$ is not defined by a closed-form formula but by properties: normalization $\int |\psi|^2 = 1$, evolution by Schrödinger equation, and boundary conditions. These properties uniquely determine $\psi$ (up to global phase).

\item \textbf{Green's Functions:} Green's functions $G(x, x')$ are defined implicitly by the differential equation $\Delta G = \delta(x - x')$ and boundary conditions, not by explicit formulas (only considering in special cases).

\item \textbf{Modular Forms:} In number theory, modular forms are defined by transformation properties under $SL(2, \mathbb{Z})$, growth conditions, and (holomorphy, not) by explicit formulas.

\item \textbf{Fixed Points in RG:} Renormalization group fixed points $g^*$ are defined implicitly by $\beta(g^*) = 0$ and stability conditions, rarely admitting closed-form solutions.

\end{enumerate}

The implicit definition of $V_{\text{div}}$ is therefore rigorous and standard. It avoids circularity with the zeta function while ensuring all necessary mathematical properties for path integral construction.

\end{remark}

\begin{remark}[Connection to Lorentzian Geometry via Wick Rotation]
\label{rem:criticalMeasureWickRotation}

The critical measure $V_{\text{div}}$ enables rigorous Wick rotation from Euclidean to Lorentzian signature (Theorem \ref{thm:wickRotation}). The polynomial bounds (P4) ensure that the complex-time extension:
\begin{equation}
V_{\text{div}}[\psi(t + i\tau)] \quad \text{for } \tau \in [0, \beta]
\end{equation}
remains uniformly bounded, allowing analytic continuation. This connection is developed further in Section \ref{sec:lorentzianGeometry}, where the Lorentzian metric emerges from the Euclidean path integral via \cite{osterwalderSchrader1973axioms} reconstruction.

\end{remark}

