% proofThmYangMillsHilbertStructure.tex
% Proof content


\begin{proof}

The proof establishes that the Bregman divergence-induced norm generates a complete Hilbert space for Yang-Mills gauge fields.

\textbf{Step 1: well-Defined Norm Structure}

For the Yang-Mills gauge field space $\Omega^1(X; \mathfrak{g})$, define:
\begin{equation}
\|\mathcal{A}\|_{\Phi}^2 := D_\Phi(\mathcal{A} \| 0) + \int_X |\Gamma(\mathcal{A}, \mathcal{A})| \, d\mu.
\end{equation}

By Lemma \ref{lem:bregmanProperties}, the Bregman divergence $D_\Phi(\cdot \| \cdot)$ is:
\begin{itemize}
\item Positive definite: $D_\Phi(\mathcal{A} \| 0) > 0$ for $\mathcal{A} \neq 0$
\item Induced by strict convexity: $V''(|\mathcal{A}|^2) \geq \lambda_0 > 0$ (Axiom \ref{ax:configSpace}(V2))
\item Measurable: $\mathcal{A} \mapsto D_\Phi(\mathcal{A} \| 0)$ is Borel measurable
\end{itemize}

The Carré du Champ term $\int_X |\Gamma(\mathcal{A}, \mathcal{A})| \, d\mu$ provides additional regularity by Theorem \ref{thm:metricFromCarre}.

\textbf{Step 2: Completeness via Cauchy Sequences}

Let $\{\mathcal{A}_n\}_{n=1}^\infty \subset L^2(X; \Lambda^1(X) \otimes \mathfrak{g})$ be Cauchy in $\|\cdot\|_\Phi$. Then:
\begin{equation}
\|\mathcal{A}_n - \mathcal{A}_m\|_\Phi^2 \to 0 \quad \text{as } n, m \to \infty.
\end{equation}

By coercivity of $\Phi$ (Axiom \ref{ax:configSpace}(V4)), the Bregman divergence satisfies:
\begin{equation}
D_\Phi(\mathcal{A}_n \| \mathcal{A}_m) \geq C_V \|\mathcal{A}_n - \mathcal{A}_m\|_{L^2}^{2\alpha / (1+\alpha)}
\end{equation}
where $\alpha > 2$ is the coercivity exponent. This implies $\{\mathcal{A}_n\}$ is Cauchy in $L^2(X; \Lambda^1 \otimes \mathfrak{g})$, so $\mathcal{A}_n \to \mathcal{A}$ strongly in $L^2$.

By lower semicontinuity of $\|\cdot\|_\Phi$ (Lemma \ref{lem:bregmanProperties}), the limit $\mathcal{A} \in \mathcal{H}_{YM}$ satisfies $\|\mathcal{A}\|_\Phi < \infty$. Thus $\mathcal{H}_{YM}$ is complete.

\textbf{Step 3: Hilbert Space Structure}

Define the inner product by polarization:
\begin{equation}
\langle \mathcal{A}, \mathcal{B} \rangle = \text{Re}(D_\Phi(\mathcal{A} \| \mathcal{B})) + \int_X \Gamma(\mathcal{A}, \mathcal{B}) \, d\mu.
\end{equation}

By bi-linearity of $\Gamma$ (Lemma \ref{lem:bregmanProperties}) and linearity of the Bregman divergence in the first variable (for fixed second variable), this defines a positive-definite sesquilinear form that satisfies the parallelogram law.

\textbf{Step 4: Separability and Spectral Condition}

The Borel regularity of $(X, d_X, \mu)$ (Axiom \ref{ax:polishSpace}) ensures that the eigenbasis $\{\phi_n\}_{n=1}^\infty$ of the Dirichlet Laplacian on $X$ is countable. The tensor product basis $\{\phi_n \otimes e_a : n \in \mathbb{N}, a \in I_G\}$ (where $I_G$ indexes the Lie algebra $\mathfrak{g}$) is countable, making $\mathcal{H}_{YM}$ separable.

The spectral condition follows from the path integral construction in Section \ref{sec:quantumPathIntegral}: the energy-momentum operator has spectrum $\text{Spec}(P^\mu) \subset V^+ = \{p : p^0 \geq |{\bf p}|, p^0 \geq 0\}$ by the Wick rotation and positivity-preserving properties of the heat kernel (Theorem \ref{thm:heatKernelBounds}).

\textbf{Step 5: Vacuum State}

The unique vacuum is the ground state of the path integral measure, given by $|0\rangle := \exp(-\Phi[\mathcal{A}_0])$ where $\mathcal{A}_0$ is the minimizer of $\Phi$. By strict convexity (Axiom \ref{ax:configSpace}(V2)), this minimizer is unique (at $\mathcal{A}_0 = 0$ after shifting), and the vacuum state is invariant under Poincaré translations and Lorentz transformations by construction of the path integral measure.

\end{proof}
