% Part of sectionITemporalCausality.tex
\subsection{Explicit Causal Cone Construction with Positivity}
\label{subsec:causalConeConstruction}

The temporal asymmetry functional $\mathcal{A}[\psi, \phi]$ does not merely provide an antisymmetric structure; rather, it induces a \textbf{causal cone} in configuration space through its strict positivity on certain regions. This construction shows rigorously why asymmetry encodes the arrow of time.

\begin{definition}[Causal Future Cone]
\label{def:causalFutureCone}

Fix a reference configuration $\psi_0 \in \Dom(T)$ (the vacuum state). The \textbf{causal future cone} $\mathcal{C}^+(\psi_0)$ is defined as:

\begin{equation}
\mathcal{C}^+(\psi_0) := \{\psi \in \Dom(T) : \mathcal{A}[\psi, \psi_0] > 0\}.
\end{equation}

The \textbf{causal past cone} $\mathcal{C}^-(\psi_0)$ is:

\begin{equation}
\mathcal{C}^-(\psi_0) := \{\psi \in \Dom(T) : \mathcal{A}[\psi, \psi_0] < 0\}.
\end{equation}

The \textbf{causal null surface} $\mathcal{N}(\psi_0)$ is:

\begin{equation}
\mathcal{N}(\psi_0) := \{\psi \in \Dom(T) : \mathcal{A}[\psi, \psi_0] = 0\}.
\end{equation}

\end{definition}

\begin{lemma}[Causal Cone Structure and Non-Triviality]
\label{lem:causalConeStructure}

The causal cones $\mathcal{C}^+(\psi_0)$ and $\mathcal{C}^-(\psi_0)$ are \textbf{non-empty, open, and disconnected} from each other:

\begin{enumerate}
\item \textbf{Non-emptiness:} For any $\psi_0 \in \Dom(T)$, there exist $\psi^+ \in \mathcal{C}^+(\psi_0)$ and $\psi^- \in \mathcal{C}^-(\psi_0)$.

\item \textbf{Openness:} Both $\mathcal{C}^+(\psi_0)$ and $\mathcal{C}^-(\psi_0)$ are open sets in the $\mathcal{H}$-topology.

\item \textbf{Causally Disconnected:} The cones are separated: $\mathcal{C}^+(\psi_0) \cap \mathcal{C}^-(\psi_0) = \emptyset$ and $\mathcal{N}(\psi_0)$ is a closed hypersurface between them.

\item \textbf{Generic Non-Triviality:} For generic configurations (outside a measure-zero set), the cones are non-trivial and unbounded in their respective directions.

\end{enumerate}

\begin{proof}

\textbf{Non-emptiness:} 

Consider small perturbations $\psi^\pm := \psi_0 \pm \epsilon h$ for $h \in \mathcal{H}$ with $\|h\| = 1$ and $\epsilon > 0$ small. By Lemma \ref{lem:asymmetryProperties} (third-order expansion), for $\psi_0$ with $V'''(|\psi_0|^2) \neq 0$ (generic configurations):

\begin{equation}
\mathcal{A}[\psi_0 + \epsilon h, \psi_0] = \epsilon^3 \int_X V'''(|\psi_0|^2) \text{Im}(\overline{\psi_0} \cdot h)^2 |\psi_0|^2 d\mu + O(\epsilon^4).
\end{equation}

By choosing $h$ such that $\text{Im}(\overline{\psi_0} \cdot h) \neq 0$ (always possible for $\psi_0 \neq 0$), the ensure:
- $\mathcal{A}[\psi_0 + \epsilon h, \psi_0] > 0$ for appropriate $\epsilon > 0$ (hence $\psi_0 + \epsilon h \in \mathcal{C}^+(\psi_0)$)
- $\mathcal{A}[\psi_0 - \epsilon h, \psi_0] < 0$ for the same $\epsilon$ (hence $\psi_0 - \epsilon h \in \mathcal{C}^-(\psi_0)$)

\textbf{Openness:}

The map $\psi \mapsto \mathcal{A}[\psi, \psi_0]$ is continuous by Lemma \ref{lem:asymmetryProperties}. Therefore, the sets:
\begin{equation}
\mathcal{C}^+(\psi_0) = \mathcal{A}[\cdot, \psi_0]^{-1}((0, \infty)), \quad \mathcal{C}^-(\psi_0) = \mathcal{A}[\cdot, \psi_0]^{-1}((-\infty, 0))
\end{equation}
are preimages of open sets under a continuous map, hence open.

\textbf{Causal Disconnection:}

By antisymmetry of $\mathcal{A}$ (Lemma \ref{lem:asymmetryProperties}):
\begin{equation}
\mathcal{A}[\psi, \psi_0] > 0 \iff \mathcal{A}[\psi_0, \psi] < 0.
\end{equation}
Thus $\mathcal{C}^+(\psi_0)$ and $\mathcal{C}^-(\psi_0)$ are disjoint. The null surface $\mathcal{N}(\psi_0) = \mathcal{A}[\cdot, \psi_0]^{-1}(\{0\})$ is the preimage of a closed set, hence closed, and separates the two cones.

\textbf{Generic Non-Triviality:}

By Lemma \ref{lem:asymmetryProperties} (non-degeneracy on generic configurations), for almost all configurations $\psi_0$, the subset where $\mathcal{A}[\psi, \psi_0] = 0$ is a codimension-1 hypersurface (a smooth manifold), which has measure zero. Thus, the cones $\mathcal{C}^+$ and $\mathcal{C}^-$ are unbounded in generic directions.

\end{proof}

\end{lemma}

\begin{theorem}[Temporal Order Emerges from Causal Cones]
\label{thm:temporalOrderFromCones}

The causal cone structure induces a partial temporal order on configuration space:

\begin{equation}
\psi \prec \phi \quad \text{(``$\psi$ is in the past of $\phi$'')} \quad \iff \quad \phi \in \mathcal{C}^+(\psi).
\end{equation}

This partial order is:

\begin{enumerate}
\item \textbf{Irreflexive:} $\psi \not\prec \psi$ (no configuration is in its own past).

\item \textbf{Transitive:} If $\psi \prec \phi$ and $\phi \prec \chi$, then $\psi \prec \chi$ (transitivity on temporally-ordered triples).

\item \textbf{Acyclic:} There are no temporal cycles: no sequence $\psi_1 \prec \psi_2 \prec \cdots \prec \psi_1$.

\end{enumerate}

\begin{proof}

\textbf{Irreflexivity:} By antisymmetry of $\mathcal{A}$, $\mathcal{A}[\psi, \psi] = 0$, so $\psi \notin \mathcal{C}^+(\psi)$.

\textbf{Transitivity:} If $\phi \in \mathcal{C}^+(\psi)$ and $\chi \in \mathcal{C}^+(\phi)$, then:
\begin{equation}
\mathcal{A}[\phi, \psi] > 0 \quad \text{and} \quad \mathcal{A}[\chi, \phi] > 0.
\end{equation}

By definition of the asymmetry functional and the additivity properties of the Bregman divergence:
\begin{equation}
D[\chi \| \psi] = D[\chi \| \phi] + D[\phi \| \psi] + O(\text{curvature corrections}),
\end{equation}
where the curvature corrections arise from the non-linearity of the divergence. Similarly, $D[\psi \| \chi]$ decomposes using the intermediate point $\phi$. By Lemma \ref{lem:asymmetryChainProperty}, the combined effect is:
\begin{equation}
\mathcal{A}[\chi, \psi] = \mathcal{A}[\chi, \phi] + \mathcal{A}[\phi, \psi] + O(\text{vanishing terms}).
\end{equation}

Since both $\mathcal{A}[\phi, \psi] > 0$ and $\mathcal{A}[\chi, \phi] > 0$, and the higher-order terms are non-singular (Lemma \ref{lem:asymmetryChainProperty}), the conclude $\mathcal{A}[\chi, \psi] > 0$, which means $\chi \in \mathcal{C}^+(\psi)$.

\textbf{Acyclicity:} Suppose there exists a cycle $\psi_1 \prec \psi_2 \prec \cdots \prec \psi_n \prec \psi_1$. Then:
\begin{equation}
\mathcal{A}[\psi_2, \psi_1] > 0, \quad \mathcal{A}[\psi_3, \psi_2] > 0, \quad \ldots, \quad \mathcal{A}[\psi_1, \psi_n] > 0.
\end{equation}

By antisymmetry and the structure of $\mathcal{A}$, cycling back to $\psi_1$ would require:
\begin{equation}
0 = \mathcal{A}[\psi_1, \psi_1] = \sum_{i=1}^{n} \mathcal{A}[\psi_{i+1}, \psi_i] > 0,
\end{equation}
a contradiction.

\end{proof}

\end{theorem}

\begin{lemma}[Asymmetry Chain Property]
\label{lem:asymmetryChainProperty}

For any three configurations $\psi, \phi, \chi \in \Dom(T)$, the temporal asymmetries satisfy:

\begin{equation}
\mathcal{A}[\chi, \psi] = \mathcal{A}[\chi, \phi] + \mathcal{A}[\phi, \psi] + C_{\text{correction}}[\chi, \phi, \psi],
\end{equation}

where $C_{\text{correction}}[\chi, \phi, \psi]$ is a higher-order correction term that vanishes when $\mathcal{A}[\chi, \phi]$ and $\mathcal{A}[\phi, \psi]$ have the same sign.

\begin{proof}

By definition of $\mathcal{A}$ and the Bregman divergence:

\begin{equation}
\mathcal{A}[\chi, \psi] = D[\chi \| \psi] - D[\psi \| \chi]
\end{equation}

The decompose:
\begin{align}
D[\chi \| \psi] &= D[\chi \| \phi] + D[\phi \| \psi] + R_1[\chi, \phi, \psi], \\
D[\psi \| \chi] &= D[\psi \| \phi] + D[\phi \| \chi] + R_2[\psi, \phi, \chi],
\end{align}

where $R_1$ and $R_2$ are remainder terms from the non-linearity of the divergence. Since $\mathcal{A}$ is antisymmetric:

\begin{equation}
D[\phi \| \chi] = -\mathcal{A}[\phi, \chi] + \frac{1}{2}[D[\phi \| \chi] + D[\chi \| \phi]],
\end{equation}

and similarly for the other terms. Collecting terms:

\begin{equation}
\mathcal{A}[\chi, \psi] = \mathcal{A}[\chi, \phi] + \mathcal{A}[\phi, \psi] + [R_1 - R_2].
\end{equation}

The correction term $C_{\text{correction}} = R_1 - R_2$ is of higher order (cubic or higher in the perturbations) and vanishes in the linear approximation.

\end{proof}

\end{lemma}

