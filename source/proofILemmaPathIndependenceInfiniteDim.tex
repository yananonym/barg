% proofLemPathIndependenceInfiniteDim.tex
% Proof content

The verify the three conditions for path-independence in infinite-dimensional Banach spaces (Lang, Differential and Riemannian Manifolds, Ch. III).

\textit{(i) Frechet Differentiability on Open Connected Domain.}

On $\Dom(T) = \{\psi : D[\psi\|\psi_0], D[\psi_0\|\psi] < \infty\}$, the functional $\mathcal{A}[\cdot, \psi_0]$ is Frechet differentiable by Definition \ref{def:temporalVectorFieldDomain} (polynomial growth of $V$ from Axiom \ref{ax:configSpace} condition V3 ensures dominated convergence applies).

The Frechet derivative is $D_\psi \mathcal{A}[\cdot, \psi_0] = T[\psi]$, which is linear and bounded in $h$.

Moreover, $\Dom(T)$ is an open subset of $\mathcal{H}$ (as the sublevel set of the lower-semicontinuous function $\max(D[\cdot\|\psi_0], D[\psi_0\|\cdot])$) and is connected (by convexity of sublevel sets of convex functions).

\textit{(ii) Simple Connectivity of Domain.}

The key observation is that $\Dom(T)$ is a convex subset of $\mathcal{H}$. To see this: the Bregman divergence $D[\psi \| \phi]$ is convex in its first argument $\psi$ (this follows from strict convexity of $V$, condition V2). Therefore:
\begin{equation}
\{\psi : D[\psi\|\psi_0] < \infty\}
\end{equation}
is a convex set (sublevel set of convex function).

Since any convex subset of a Banach space is simply connected, $\Dom(T)$ is simply connected. This eliminates topological obstructions to path-independence.

\textit{(iii) Fundamental Theorem of Calculus in Banach Spaces.}

For any $C^1$ path $\gamma: [0,1] \to \Dom(T)$ (i.e., differentiable with continuous derivative), the fundamental theorem of calculus for Banach spaces (Lang 1995, Dieudonne 1969) states:

\begin{equation}
f(\gamma(1)) - f(\gamma(0)) = \int_0^1 \langle D f[\gamma(s)], \gamma'(s) \rangle ds,
\end{equation}
where $f$ is Frechet differentiable and the inner product is taken between the Frechet derivative (in the dual space $\mathcal{H}^*$) and the tangent vector (in $\mathcal{H}$).

Applying this with $f = \mathcal{A}[\cdot, \psi_0]$:
\begin{equation}
\mathcal{A}[\gamma(1), \psi_0] - \mathcal{A}[\gamma(0), \psi_0] = \int_0^1 \langle T[\gamma(s)], \gamma'(s) \rangle ds,
\end{equation}
where $T[\gamma(s)] = D_\psi \mathcal{A}[\gamma(s), \psi_0]$ from Definition \ref{def:temporalVectorFieldDomain}.

\textit{(iv) Path-Independence Conclusion.}

Since the right-hand side depends only on the starting point $\gamma(0)$, the ending point $\gamma(1)$, and the value of $\mathcal{A}[\cdot, \psi_0]$ (not on the path itself), there is:

For any two paths $\gamma_1, \gamma_2$ connecting $\phi$ to $\psi$:
\begin{equation}
\int_0^1 \langle T[\gamma_1(s)], \gamma_1'(s) \rangle ds = \mathcal{A}[\psi, \psi_0] - \mathcal{A}[\phi, \psi_0] = \int_0^1 \langle T[\gamma_2(s)], \gamma_2'(s) \rangle ds.
\end{equation}

Therefore, the line integral $\int \langle T, d\gamma \rangle$ is independent of the path, and the temporal potential $t[\psi] := \mathcal{A}[\psi, \psi_0] - \mathcal{A}[\psi_0, \psi_0] = \mathcal{A}[\psi, \psi_0]$ (since $\mathcal{A}[\psi_0, \psi_0] = 0$) is a well-defined scalar function on $\Dom(T)$. \qed
