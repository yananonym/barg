\subsection{Temporal Structure and Foliations}

\begin{lemma}[Temporal Vector Field from Asymmetry]
\label{lem:temporalVectorFromAsymmetry}
The divergence asymmetry functional $\mathcal{A}[\psi](x) = D[\psi \| \psi_0] - D[\psi_0 \| \psi]$ has a well-defined functional gradient on the emerged manifold $\mathcal{M}$:
\[
\nabla \mathcal{A}(x) \in T_x^* \mathcal{M}
\]

This gradient defines a non-vanishing vector field $T^\mu(x) := g^{\mu\nu}(x) \nabla_\nu \mathcal{A}(x)$ that is nowhere null with respect to the Riemannian metric $g$.

\textit{Significance:} This vector field selects a distinguished temporal direction, emerging from the asymmetry of the Bregman divergence and the structure of the generating functional $\Phi$. All external time-like direction is imposed; instead, it emerges from the information-theoretic foundation.

\begin{proof}
The functional derivative $\delta \mathcal{A}/\delta \psi$ exists in $L^2(X)$ by condition V1 ($V'(0) = 0$). The gradient is:
\[
\nabla \mathcal{A}(x) = \left(\delta \mathcal{A}/\delta \psi\right)|_x = \int_X G(x,y) [V'(|\psi(y)|^2)\psi(y) - V'(|\psi_0(y)|^2)\psi_0(y)] d\mu(y)
\]
where $G$ is the Green function of the Laplacian $A$. This is well-defined in $L^2$ and defines a measurable vector field. By Lemma \ref{thm:lapsePositivity}, $|T(x)|^2 = g_{\mu\nu} T^\mu T^\nu > 0$ almost everywhere. \qed
\end{proof}
\end{lemma}% proof_lem_asymmetry_to_lorentzian_signature.tex

\begin{lemma}[Bregman Divergence Asymmetry Induces Temporal Direction]
\label{lem:temporalDirectionFromAsymmetry}

The asymmetry of the Bregman divergence,
\begin{equation}
\mathcal{A}[\psi](x) := D_\Phi[\psi \| \psi_0](x) - D_\Phi[\psi_0 \| \psi](x),
\end{equation}
defines a distinguished temporal direction on the manifold $\mathcal{M}$ via the functional gradient:
\begin{equation}
T^\mu(x) := g^{\mu\nu}(x) \nabla_\nu \mathcal{A}(x),
\end{equation}
where $g_{\mu\nu}$ is the emergent Riemannian metric from the Carre du Champ operator (Theorem \ref{thm:metricFromCarre}).

This vector field has the following properties:

\begin{enumerate}

\item \textbf{well-Definedness:} $T^\mu(x)$ is well-defined and nowhere null with respect to the Riemannian metric $g$:
\begin{equation}
g_{\mu\nu} T^\mu T^\nu > 0 \quad \mu\text{-a.e. on } X.
\end{equation}

\item \textbf{Monotonicity (Lyapunov Property):} The functional $\mathcal{A}[\psi]$ is non-decreasing along RG flow trajectories in a neighborhood of the fixed point $g^*$.

\item \textbf{Uniqueness:} The temporal direction $T$ is unique up to positive scaling (multiplication by positive functions).

\end{enumerate}

\begin{proof}

\textbf{Step 1: Asymmetry Definition}

The Bregman divergence is defined as:
\begin{equation}
D_\Phi[\psi_1 \| \psi_2] := \Phi[\psi_1] - \Phi[\psi_2] - \langle D\Phi[\psi_2], \psi_1 - \psi_2 \rangle.
\end{equation}

By strict convexity of $\Phi$ (Axiom II), $D_\Phi[\psi_1 \| \psi_2] \geq 0$ with equality if and only if $\psi_1 = \psi_2$. However:
\begin{equation}
D_\Phi[\psi_1 \| \psi_2] \neq D_\Phi[\psi_2 \| \psi_1] \text{ in general}.
\end{equation}

The asymmetry functional is:
\begin{equation}
\mathcal{A}[\psi] := D_\Phi[\psi \| \psi_*] - D_\Phi[\psi_* \| \psi],
\end{equation}
where $\psi_*$ is a reference configuration (e.g., the vacuum state or critical surface).

\textbf{Step 2: Gradient as Temporal Direction}

The functional gradient of $\mathcal{A}$ is:
\begin{equation}
\frac{\delta \mathcal{A}}{\delta \psi}(x) = 2[V'(|\psi_0|^2) - V'(|\psi|^2)] + \text{higher-order terms}.
\end{equation}

This functional gradient is a function on $X$. Via the Riesz representation theorem on the Hilbert space $\mathcal{H} = L^2(X; \mathbb{C}^n)$, this defines an element $\delta \mathcal{A}/\delta \psi \in L^2(X; \mathbb{C}^n)$.

The metric gradient (using the emergent Riemannian metric):
\begin{equation}
T^\mu := g^{\mu\nu} \nabla_\nu \mathcal{A}
\end{equation}
is a tangent vector to the manifold $\mathcal{M}$, i.e., a vector field in the tangent bundle $T\mathcal{M}$.

\textbf{Step 3: Non-Nullness}

The verify that $T$ is not null with respect to the Riemannian metric $g$. By strict convexity of $\Phi$, the divergence $D_\Phi[\psi \| \psi_0] > 0$ for $\psi \neq \psi_0$. Therefore, the gradient of $\mathcal{A}$ is non-vanishing (only considering at critical points), implying:
\begin{equation}
|T|^2 = g_{\mu\nu} T^\mu T^\nu = |g(\nabla \mathcal{A}, \nabla \mathcal{A})| > 0 \quad \mu\text{-a.e.}
\end{equation}

\textbf{Step 4: Monotonicity (Lyapunov Property)}

Under the RG flow $\psi(k)$ with beta function $\beta(g) = -\lambda(g) g^{ij}(g) \partial W/\partial g_j$, the divergence asymmetry satisfies:
\begin{equation}
\frac{d\mathcal{A}[\psi(k)]}{dk} = \left\langle \frac{\delta \mathcal{A}}{\delta \psi}, \beta(\psi) \right\rangle \geq 0 \quad \text{near the fixed point},
\end{equation}
with equality if and only if $\psi = \psi_*$. This is the Lyapunov property: $\mathcal{A}$ is a Lyapunov function for the RG flow, decreasing toward the critical surface.

\end{proof}

\end{lemma}

\begin{lemma}[\cite{osterwalderSchrader1973axioms} Axioms Hold at the Fixed Point]
\label{lem:osAxiomsFromDivergence}

Given the temporal foliation $\mathcal{M} = \bigcup_{\tau \in \mathbb{R}} \Sigma_\tau$ induced by the temporal vector field $T^\mu$ from Lemma \ref{lem:temporalDirectionFromAsymmetry}, the Euclidean field theory defined by the Dirichlet form action:
\begin{equation}
S_E[\psi] = \int_0^\beta d\tau \int_X \left[\frac{1}{2}|\partial_\tau \psi|^2 + \mathcal{E}(\psi, \psi)\right] d\mu + \int_X V(|\psi|^2) d\mu
\end{equation}
satisfies all four \cite{osterwalderSchrader1973axioms} axioms:

\begin{enumerate}

\item \textbf{OS0 (Euclidean Invariance):} The action $S_E[\psi]$ is invariant under rotations in the spatial directions and translations in the temporal direction, up to boundary terms absorbed in the functional measure.

\item \textbf{OS1 (Tempered Distributions / Regularity):} The n-point correlation functions:
\begin{equation}
G_E(x_1, \ldots, x_n) := \langle \psi(x_1) \cdots \psi(x_n) \rangle_E
\end{equation}
are tempered distributions (polynomially bounded) with respect to the Euclidean metric.

\item \textbf{OS2 (Reflection Positivity):} For the temporal reflection $\Theta_t: (t, \vec{x}) \mapsto (-t, \vec{x})$, the sesquilinear form:
\begin{equation}
\langle \mathcal{A}_+ \phi, \mathcal{A}_- \psi \rangle := \int d\mu_E[\phi_+] d\mu_E[\phi_-] \mathcal{O}_1[\phi_+] \Theta_t(\mathcal{O}_2[\phi_-])
\end{equation}
is non-negative for all local observables $\mathcal{O}_1, \mathcal{O}_2$ and all field configurations $\phi_\pm$.

\item \textbf{OS3 (Cluster Property):} For large Euclidean separations, two-point functions decay exponentially:
\begin{equation}
|\langle \psi(x) \psi(y) \rangle_E - \langle \psi \rangle^2| \lesssim e^{-m |x-y|}, \quad m > 0,
\end{equation}
where $m$ is the mass gap (first excited state energy above the vacuum).

\end{enumerate}

\begin{proof}

\textbf{Verification of OS0:} By Theorem \ref{thm:metricFromCarre}, the emergent metric $g_{\mu\nu}$ is positive definite on $\mathcal{M}$, and the Dirichlet form $\mathcal{E}(\psi, \psi)$ is the integral of the quadratic form associated with $g$:
\begin{equation}
\mathcal{E}(\psi, \psi) = \int_X g^{\mu\nu} (\partial_\mu \psi) (\partial_\nu \psi) d\mu.
\end{equation}

Under rotations $\text{SO}(Q) \ni R$ in spatial directions, this quadratic form is preserved. Under temporal translations by the vector field $T$, the action is invariant due to time-translational symmetry of the Lagrangian.

\textbf{Verification of OS1:} The path integral measure $\mathcal{D}\psi$ (Theorem \ref{thm:pathIntegralConstruction}) is a Gaussian measure (up to the potential $V$), and the Gaussian measure has tempered distribution properties. The correlation functions $G_E$ satisfy polynomial bounds from the Wick theorem and the Gaussian measure properties.

\textbf{Verification of OS2 (Reflection Positivity):} This is the key axiom relating to Bregman asymmetry. The Bregman divergence's asymmetry ensures that the "past" ($t < 0$) and "future" ($t > 0$) are distinguished. By the log-concavity of the measure (from the strict convexity of $\Phi$, Lemma \ref{lem:bregmanProperties}), the measure $d\mu_E \propto \exp(-S_E[\psi])$ satisfies the FKG (Fortuin-Kasteleyn-Ginibre) inequality, which implies reflection positivity (Theorem 3.3 of \cite{osterwalderSchrader1973axioms} 1973).

\textbf{Verification of OS3 (Clustering):} The spectral gap of the Hamiltonian $H = -\Delta + V''(|\psi_0|^2)$ is strictly positive (Lemma \ref{lem:spectralGapComplete}). This gap translates to exponential clustering of correlation functions:
\begin{equation}
\langle \psi(x) \psi(y) \rangle_E = \int_0^\infty d\lambda \, e^{-\lambda |x-y|} \rho(\lambda),
\end{equation}
where $\rho(\lambda)$ is the spectral measure with $\rho(\lambda) = 0$ for $\lambda < m$ (by the mass gap).

\end{proof}

\end{lemma}

\begin{theorem}[Wick Rotation and Lorentzian Signature from Asymmetry]
\label{thm:wickRotationRigorous}

Under the divergence-first framework with the temporal direction $T^\mu$ from Lemma \ref{lem:temporalDirectionFromAsymmetry}, and given that all four \cite{osterwalderSchrader1973axioms} axioms hold (Lemma \ref{lem:osAxiomsFromDivergence}), the Wick rotation:
\begin{equation}
\tau = it \quad (\text{Euclidean time} \to \text{Lorentzian time})
\end{equation}
yields a unique Lorentzian QFT with signature $(-,+,+,+)$.

\textbf{Analyticity Domain:} The Wick rotation is valid throughout the weak-coupling regime $|g_s| < g_{\text{crit}}$ where the mass gap $\Delta(g_s) > 0$ is strictly positive. The spectral gap is essential for Lehmann analyticity of two-point functions, which in turn guarantees that the Euclidean correlators extend analytically to the Minkowski domain and validates the analytic continuation procedure.

\begin{proof}

By the \cite{osterwalderSchrader1973axioms} reconstruction theorem (Osterwalder and Schrader 1973, 1975), a Euclidean field theory satisfying all four OS axioms can be uniquely analytically continued to a Lorentzian field theory with the following properties:

\begin{enumerate}

\item \textbf{Analyticity:} The n-point Euclidean correlation functions extend analytically to the tube domain $\mathcal{T}_n = \{(\zeta_1, \ldots, \zeta_n) \in \mathbb{C}^n : \text{Im}(\zeta_1) > \cdots > \text{Im}(\zeta_n)\}$.

\item \textbf{Analytic Continuation (Wick Rotation):} Replacing $\tau_j \to -it_j$ with $t_j \in \mathbb{R}$, the Lorentzian n-point functions are:
\begin{equation}
G_L(t_1, \vec{x}_1; \ldots; t_n, \vec{x}_n) := G_E(-it_1, \vec{x}_1; \ldots; -it_n, \vec{x}_n).
\end{equation}

\item \textbf{Signature Determination:} The reflection positivity axiom (OS2) determines the metric signature. Specifically, the time reflection symmetry breaking induced by the Bregman asymmetry forces a change of sign in the time metric:
\begin{equation}
ds_E^2 = d\tau^2 + h_{ij} dx^i dx^j \quad \Rightarrow \quad ds_L^2 = -dt^2 + h_{ij} dx^i dx^j,
\end{equation}
yielding signature $(-,+,+,+)$.

\item \textbf{Wightman Axioms:} The resulting Lorentzian QFT automatically satisfies the Wightman axioms, including:
  - Poincaré covariance
  - Spectral condition (energies in forward light cone)
  - Microcausality
  - Positive definite inner product

\item \textbf{Wightman Axioms:} The resulting Lorentzian QFT automatically satisfies the Wightman axioms, including:
  - Poincaré inequality covariance
  - Spectral condition (energies in forward light cone)
  - Microcausality
  - Positive definite inner product

\item \textbf{Mass Gap and Lehmann Analyticity:} The mass gap $\Delta(g_s) > 0$ (established in Theorem \ref{thm:yangMillsComplete}) ensures exponential decay of Euclidean correlators:
\begin{equation}
G_E(\vec{x}, \tau) \sim \exp(-\Delta(g_s) |\tau|) \quad \text{as } |\tau| \to \infty.
\end{equation}
This exponential decay is the crucial property ensuring Lehmann analyticity: the two-point function admits a spectral representation valid for all complex frequencies, which is necessary for the analytic continuation from Euclidean to Minkowski spacetime.

\end{enumerate}

\textbf{Uniqueness:} The \cite{osterwalderSchrader1973axioms} reconstruction is unique: there is exactly one Lorentzian QFT corresponding to a given Euclidean theory satisfying OS axioms. All additional choices or ambiguities arise.

\end{proof}

\end{theorem}

\begin{remark}[Physical Significance of Divergence Asymmetry]
\label{rem:physicalsignificanceofdivergenceasymmetry}

The asymmetry of the Bregman divergence is not merely a mathematical artifact. It encodes the fundamental directionality of time in quantum field theory. Without the asymmetry $D[\psi \| \phi] \neq D[\phi \| \psi]$, the temporal direction would be undefined, and the Wick rotation would not yield a unique Lorentzian signature.

in the divergence-first framework, this asymmetry arises from the strict convexity of the generating functional $\Phi$ (Axiom II, Component II.ii), which is the fundamental axiom governing the dynamics of the theory. The temporal structure of spacetime thus emerges necessarily from the information-geometric properties of the configuration space, not from external assumptions about time or causality.

\end{remark}

% =========================================================================
% GRAVITON PREDICTIONS FROM D=4 CONFORMAL STRUCTURE
% =========================================================================

