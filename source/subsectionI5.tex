% Part of sectionITemporalCausality.tex
\subsection{Causal Structure and Temporal Ordering}

\begin{definition}[Causal Ordering on Configuration Space]
\label{def:pathCausalOrdering}

For $\phi, \psi \in \Dom(T)$ (the domain of the temporal vector field $T[\cdot]$ from Definition \ref{def:temporalVectorFieldDomain}), Define a causal relation $\phi \prec \psi$ if there exists a continuous path $\gamma: [0,1] \to \Dom(T)$ with $\gamma(0) = \phi$ and $\gamma(1) = \psi$ such that:
\begin{equation}
\int_0^1 \langle T[\gamma(s)], \gamma'(s) \rangle_{\mathcal{H}} ds > 0,
\end{equation}
where $\langle \cdot, \cdot \rangle_{\mathcal{H}}$ is the $L^2(X)$ inner product and the path derivative $\gamma'(s)$ is interpreted in the weak sense.
\end{definition}

\begin{lemma}[Causality is Acyclic]
\label{lem:causalityAcyclic}

The relation $\prec$ (Definition \ref{def:pathCausalOrdering}) is acyclic: there is no configuration $\psi$ such that $\psi \prec \psi$.

\begin{proof}

Suppose toward contradiction that $\psi \prec \psi$. Then there exists a closed path $\gamma: [0,1] \to \Dom(T)$ with $\gamma(0) = \gamma(1) = \psi$ such that:
\begin{equation}
\oint_\gamma \langle T[\gamma(s)], \gamma'(s) \rangle_{\mathcal{H}} ds > 0.
\end{equation}

By the divergence theorem on the path space (or equivalently, integration by parts), for a closed loop:
\begin{equation}
\oint_\gamma \langle T[\gamma(s)], d\gamma(s) \rangle = \int_{\text{interior}} \mathrm{div}(T) \, d\mu = 0
\end{equation}
if $T$ is a conservative vector field (derived from the temporal functional $\phi_0$). Since $T[\psi] = \nabla_{\min} \phi_0(\psi)$ is indeed a gradient field (by definition, Theorem \ref{thm:lapsePositivityFromDivergence}), the integral around any closed loop must vanish.

This contradicts the positivity assumption. Therefore, no closed loop exists with positive integral, so $\psi \not\prec \psi$. \qed

\end{proof}

\end{lemma}

\begin{lemma}[Causality is Transitive]
\label{lem:causalityTransitive}

If $\phi \prec \psi$ and $\psi \prec \chi$, then $\phi \prec \chi$.

\begin{proof}

By definition of $\prec$, there exist paths $\gamma_1: [0,1] \to \Dom(T)$ with $\gamma_1(0) = \phi$, $\gamma_1(1) = \psi$ and:
\begin{equation}
\Delta_1 := \int_0^1 \langle T[\gamma_1(s)], \gamma_1'(s) \rangle ds > 0,
\end{equation}
and $\gamma_2: [0,1] \to \Dom(T)$ with $\gamma_2(0) = \psi$, $\gamma_2(1) = \chi$ and:
\begin{equation}
\Delta_2 := \int_0^1 \langle T[\gamma_2(s)], \gamma_2'(s) \rangle ds > 0.
\end{equation}

Concatenate the paths: define $\gamma: [0,2] \to \Dom(T)$ by:
\begin{equation}
\gamma(s) := \begin{cases} \gamma_1(2s) & \text{for } s \in [0, 1/2] \\ \gamma_2(2s-1) & \text{for } s \in [1/2, 1] \end{cases}
\end{equation}

(Reparameterize to $[0,1]$ via $\tilde{\gamma}(s) = \gamma(2s)$ for $s \in [0,1]$.)

The concatenated path satisfies $\gamma(0) = \phi$ and $\gamma(1) = \chi$. The integral becomes:
\begin{equation}
\int_0^1 \langle T[\gamma(s)], \gamma'(s) \rangle ds = \Delta_1 + \Delta_2 > 0.
\end{equation}

Thus $\phi \prec \chi$ via the concatenated path. \qed

\end{proof}

\end{lemma}

\begin{theorem}[Temporal Ordering is a Strict Partial Order]
\label{thm:causalityPartialOrder}

The relation $\prec$ defines a strict partial order on $\Dom(T)$: it is irreflexive (Lemma \ref{lem:causalityAcyclic}), transitive (Lemma \ref{lem:causalityTransitive}), and asymmetric (i.e., $\psi \prec \phi$ implies $\phi \not\prec \psi$).

The causal structure induces a partial temporal ordering on configuration space, with configurations partitioned into causally disconnected layers. This provides a rigorous foundation for causality in the divergence-first framework.

\begin{proof}

\noindent\textbf{Irreflexivity:} By Lemma \ref{lem:causalityAcyclic}, $\psi \not\prec \psi$ for all $\psi \in \Dom(T)$.

\noindent\textbf{Transitivity:} By Lemma \ref{lem:causalityTransitive}, if $\phi \prec \psi$ and $\psi \prec \chi$, then $\phi \prec \chi$.

\noindent\textbf{Asymmetry:} Suppose $\psi \prec \phi$. Then there exists a path $\gamma$ with positive temporal integral from $\psi$ to $\phi$. If also $\phi \prec \psi$, then the reversed path would have positive integral from $\phi$ to $\psi$, implying a positive integral around the loop, contradicting Lemma \ref{lem:causalityAcyclic}. Therefore, $\phi \not\prec \psi$.

These three properties define a strict partial order.

\end{proof}

\end{theorem}
