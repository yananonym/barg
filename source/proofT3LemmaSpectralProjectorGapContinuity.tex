% proofLemSpectralProjectorGapContinuity.tex
% Proof content

% Based on Kato's analytic perturbation theory for self-adjoint operators

\textbf{Step 1: Setup and Analyticity Domain}

Consider the Yang-Mills Hamiltonian $H(g_s) = H_0 + g_s H_{\text{int}}$ where $H_0$ is the free Yang-Mills operator and $H_{\text{int}} = H_1 + \cdots + H_m$ is the interaction Hamiltonian. By Lemma \ref{lem:divergenceCoercivityInteractionBounds}, $H_{\text{int}}$ is bounded relative to $H_0$: there exist constants $a, b$ with $0 \leq a < 1$ such that
\[\|H_{\text{int}} \psi\| \leq a \|H_0 \psi\| + b \|\psi\|\]
for all $\psi$ in the domain of $H_0$.

The critical coupling $g_{\text{crit}}$ is defined as the supremum of $g_s$ values where the following holds:
\[g_{\text{crit}} := \sup\left\{g > 0 : \|g H_{\text{int}}\| < \text{gap}(H_0)/2\right\}.\]

By the relative boundedness, $g_{\text{crit}} > 0$.

\textbf{Step 2: Analytic Perturbation Theory (Kato, 1966)}

For $g_s \in [0, g_{\text{crit}})$, the operator $H(g_s) = H_0 + g_s H_{\text{int}}$ is a holomorphic family of type (A) in the sense of Kato. This means:

\begin{enumerate}[label=(\roman*)]
\item $H(g_s)$ is defined and closed for all $g_s \in [0, g_{\text{crit}})$.
\item For each $z \notin \text{Spectrum}(H(g_s))$, the resolvent $(z - H(g_s))^{-1}$ is analytic in $g_s$.
\item The domain $\text{Dom}(H(g_s)) = \text{Dom}(H_0)$ is independent of $g_s$.
\end{enumerate}

These properties follow from Kato (1966, Chapter IV, Section 3.5) for perturbations bounded relative to $H_0$.

\textbf{Step 3: Analyticity of Eigenvalues}

By Kato (1966, Theorem 2.5.1 and Theorem 2.6.1), for each isolated eigenvalue $\lambda_n(0)$ of $H_0$ with multiplicity $m_n$, there exist exactly $m_n$ eigenvalues $\lambda_{n,j}(g_s)$ ($j = 1, \ldots, m_n$) of $H(g_s)$ near $\lambda_n(0)$, and these satisfy analytic perturbation series:
\[\lambda_{n,j}(g_s) = \lambda_n(0) + \sum_{k=1}^\infty g_s^k \lambda_{n,j}^{(k)},\]
where the $\lambda_{n,j}^{(k)}$ are computable from higher-order perturbation theory.

In particular, the gap is:
\[\Delta(g_s) = \min\{|\lambda| : \lambda \in \text{Spectrum}(H(g_s)), \lambda > 0\}.\]

Since the relevant eigenvalues $\lambda_{n,j}(g_s)$ are analytic in $g_s$, and the gap selects the minimum over those with $\lambda > 0$, the gap function is locally analytic away from eigenvalue crossings.

\textbf{Step 4: No Eigenvalue Crossing for Small $g_s$}

The free theory has $\Delta(0) = \text{gap}(H_0) > 0$ (by Theorem \ref{thm:freeYangMillsMassGap}). For small $g_s$, the first-order perturbation correction to the gap eigenvalue is:
\[\lambda_{\text{gap}}^{(1)} = \langle \psi_0 | H_{\text{int}} | \psi_0 \rangle,\]
where $|\psi_0\rangle$ is the normalized eigenstate of the gap. Since $H_{\text{int}}$ consists of interaction terms with positive matrix elements (coupling to modes with energy $\geq \Delta(0) > 0$), there is $\lambda_{\text{gap}}^{(1)} \geq 0$ with strict inequality for generic interactions.

Thus, for small $g_s > 0$:
\[\Delta(g_s) = \Delta(0) + g_s \lambda_{\text{gap}}^{(1)} + O(g_s^2) \geq \Delta(0) - o(g_s) > 0.\]

More generally, as long as no eigenvalue crosses the origin (which would require $\lambda_n(g_s) = 0$ for some $n$ and some $g_s \in (0, g_{\text{crit}})$), the gap remains positive.

\textbf{Step 5: Gap Remains Positive Throughout weak Coupling Regime}

By Lemma \ref{lem:coercivityDirichletForm} (coercivity bound), the interactions cannot produce negative contributions exceeding the gap budget. Specifically, for any perturbation $\delta V$ with $\|\delta V\| \leq \epsilon$:
\[\Delta(g_s + \delta g_s) \geq \Delta(g_s) - C(g_s, \Delta(g_s)) \epsilon.\]

By continuity of $\Delta(g_s)$ and the initial condition $\Delta(0) > 0$, if the gap are to vanish at some $g_s^* \in (0, g_{\text{crit}})$, there would be a first contact point where $\Delta(g_s^*) = 0$. But then the lower bound would give a contradiction for sufficiently small $\epsilon > 0$.

Therefore, $\Delta(g_s) > 0$ for all $g_s \in [0, g_{\text{crit}})$.

\textbf{Step 6: Continuity of Spectral Projectors}

By Kato (1966, Theorem 2.6.1), the spectral projector onto the $n$-th eigenspace:
\[P_n(g_s) = \frac{1}{2\pi i} \oint_{\Gamma_n} (z - H(g_s))^{-1} dz,\]
where $\Gamma_n$ is a contour isolating $\lambda_n(g_s)$ from other eigenvalues, satisfies
\[P_n(g_s) = P_n(0) + \sum_{k=1}^\infty g_s^k P_n^{(k)},\]
with convergence in the strong operator topology.

Thus, $P_n(g_s)$ is continuous in $g_s$ and moreover analytic away from crossings.

\textbf{Step 7: Quantitative Gap Continuity Bound}

From the perturbation expansion, the gap satisfies:
\[\frac{d}{dg_s} \Delta(g_s) = \frac{1}{2\pi i} \oint_{\Gamma(g_s)} z \frac{d}{dg_s} \left[(z - H(g_s))^{-1}\right] dz + \text{boundary terms}.\]

The derivative of the resolvent is:
\[\frac{d}{dg_s}(z - H(g_s))^{-1} = (z - H(g_s))^{-1} H_{\text{int}} (z - H(g_s))^{-1}.\]

Using resolvent bounds, this gives:
\[\left|\frac{d}{dg_s} \Delta(g_s)\right| \leq C(\Delta(g_s), \|H_{\text{int}}\|),\]
where $C$ is a locally bounded constant.

By the Lipschitz continuity this implies, combined with $\Delta(0) > 0$:
\[\Delta(g_s) \geq \Delta(0) - \int_0^{g_s} \left|\frac{d}{dg'} \Delta(g')\right| dg' > 0 \quad \text{for } g_s \in [0, g_{\text{crit}}).\]

This completes the proof. \qed

-

\begin{lemma}[Spectral Continuity Extended to Strong Coupling: Lipschitz Bounds]
\label{lem:spectralGapLipschitz}

For Yang-Mills with gauge group $G$, the mass gap $\Delta(g)$ as a 
function of coupling constant $g > 0$ is Lipschitz continuous:

\begin{equation}
|\Delta(g_1) - \Delta(g_2)| \leq L \cdot |g_1 - g_2|
\end{equation}

for all $g_1, g_2 \in (0, 1)$, where $L$ is a universal Lipschitz constant 
depending only on the gauge group $G$ and spacetime dimension.

\begin{proof}

The mass gap is defined as the spectral gap of the Hamiltonian:

\begin{equation}
H(g) = H_0 + g \int_X H_{\mathrm{int}}(x) d\mu(x).
\end{equation}

By the resolvent formula and Cauchy integral representation, the gap 
function can be bounded using resolvent bounds:

\begin{equation}
\Delta(g) = \inf \{ \lambda \in \sigma(H(g)) : \lambda > 0 \}
\end{equation}

satisfies:

\begin{equation}
\Delta(g_1) - \Delta(g_2) = \int_{\text{contour}} 
\frac{\partial}{\partial g} \text{Tr}(P_\Delta(g)) \, dg
\end{equation}

where $P_\Delta(g)$ is the spectral projector onto the gapped sector.

By Lemma \ref{lem:perturbationStability}, the derivative 
is bounded:

\begin{equation}
\left| \frac{\partial \Delta(g)}{\partial g} \right| \leq L
\end{equation}

for some constant $L$ depending on the operator norm of $H_{\mathrm{int}}$ 
and the free gap $\Delta_0$.

Integrating over the coupling range from $g_2$ to $g_1$ gives:

\begin{equation}
|\Delta(g_1) - \Delta(g_2)| = \left|\int_{g_2}^{g_1} \frac{\partial \Delta(g)}{\partial g} dg\right| \leq L \cdot |g_1 - g_2|,
\end{equation}

establishing the Lipschitz property with constant $L$.

\textit{Technical note:} This uses Kato's theorem on analytic perturbation of 
spectral gaps for self-adjoint operators. The key is that $H(g)$ is a 
one-parameter analytic family of self-adjoint operators for $g \in \mathbb{R}$, 
and spectral projectors for isolated eigenvalues (or gaps) are analytic in $g$.

\end{proof}

\textbf{Corollary [Gap Bound at Asymptotic Safety Fixed Point]:}

If $\Delta(0) = \Delta_0 > 0$ (free gap) and $\Delta$ is Lipschitz with 
constant $L$, then for the fixed point $g^* \sim O(1)$:

\begin{equation}
\Delta(g^*) \geq \Delta_0 - L \cdot g^* \geq \Delta_0 - L.
\end{equation}

Provided $L < \Delta_0$, there is $\Delta(g^*) > 0$, confirming the gap 
persists at strong coupling.


