% proofYLemmaFieldStrengthAsymmetry.tex

The field strength defined via the cyclic defect of the Bregman divergence coincides with the standard Yang-Mills field strength (curvature of the Berry connection) up to higher-order terms.

\textbf{Step 1: Cyclic Defect and Local Transport.}

Consider three infinitesimally close reference measures $\mu_t$, $\mu_{t+dt}$, $\mu_{t+dt+d't}$ parametrized by manifold coordinates. The cyclic defect is:
\begin{equation}
\Delta D := D_\Phi(\mu_t \| \mu_{t+dt}) + D_\Phi(\mu_{t+dt} \| \mu_{t+dt+d't}) + D_\Phi(\mu_{t+dt+d't} \| \mu_t).
\end{equation}

By the definition of Bregman divergence (Definition \ref{def:bregman}):
\begin{equation}
D_\Phi(\psi \| \phi) = \Phi[\psi] - \Phi[\phi] - \langle D\Phi[\phi], \psi - \phi \rangle.
\end{equation}

Expanding to second order in $dt$ and $d't$:
\begin{align}
D_\Phi(\mu_t \| \mu_{t+dt}) &= \frac{1}{2} \langle H_\Phi(t)[dt], dt \rangle + O(|dt|^3),
\end{align}
where $H_\Phi(t)$ is the Hessian of $\Phi$ at $\mu_t$.

\textbf{Step 2: Cyclic Sum and Noncommutativity.}

When the cycle around the three points, the Hessians at different locations contribute. The key observation is that the Hessian is not constant across the manifold; it varies with the reference measure:
\begin{equation}
H_\Phi(t + dt) = H_\Phi(t) + \nabla H_\Phi \cdot dt + O(|dt|^2).
\end{equation}

The cyclic sum becomes sensitive to the \textit{noncommutativity} of Hessian variations. Specifically, the cyclic defect is:
\begin{equation}
\Delta D = \langle [\nabla_\mu H_\Phi, \nabla_\nu H_\Phi] \, dt^\mu d't^\nu, \cdot \rangle + O(|dt|^3, |d't|^3),
\end{equation}
where the commutator $[\cdot, \cdot]$ denotes the Lie bracket of derivative operators.

\textbf{Step 3: Connection to Eigenspace Mixing.}

The Hessian $H_\Phi$ can be diagonalized by the eigenfunctions $\{e_n\}$ of the Laplacian $\Delta$. When the reference measure varies, the eigenbasis must adapt. The adaptation is captured by the Berry connection:
\begin{equation}
A_\mu^{ab} = -i \langle e_a, \nabla_\mu e_b \rangle.
\end{equation}

The curvature of this connection is:
\begin{equation}
F_{\mu\nu}^{ab} = \partial_\mu A_\nu^{ab} - \partial_\nu A_\mu^{ab} + [A_\mu, A_\nu]^{ab}.
\end{equation}

\textbf{Step 4: Equivalence of Definitions.}

The noncommutativity of Hessian variations (which defines the cyclic defect) is precisely encoded in the curvature $F_{\mu\nu}$ of the Berry connection. Specifically, the eigenspace mixing (captured by $A$) induces changes in the Hessian matrix elements as one varies the reference measure, and the cyclic deficit is:
\begin{equation}
\Delta D = \langle \text{Tr}(F_{\mu\nu}F^{\mu\nu}) \, dt^\mu d't^\nu, \cdot \rangle + \text{higher-order terms}.
\end{equation}

The proportionality between the cyclic defect and the field strength squared ($F_{\mu\nu}F^{\mu\nu}$) reflects the fact that the field strength measures the obstruction to having a globally-defined eigenbasis.

\textbf{Step 5: Error Terms.}

The error terms come from:
\begin{enumerate}
\item Third-order variations: $O(|dt|^3, |d't|^3, |dt||d't|^2)$ from higher-order Hessian expansions.
\item Curvature corrections: Second-order curvature terms in the Riemannian geometry, of order $O(R_{\mu\nu})$.
\item Gauge-curvature interactions: Mixed terms involving both spacetime and gauge curvatures.
\end{enumerate}

These are all $O(\|\psi - \phi\|_{\mathcal{H}}^3)$ as required.

\textbf{Conclusion.} The cyclic defect of the Bregman divergence and the curvature $F_{\mu\nu}$ of the Berry connection are equivalent definitions of the gauge field strength, differing only by terms that vanish in the infinitesimal limit. This establishes that the Yang-Mills field arises geometrically from the asymmetry of the divergence.

\end{document}
