% proofLemSpatialMetricPositivity.tex
% Self-Contained Proof via Spectral Properties (No Forward References)

\noindent\textbf{Metric Positivity via Carré du Champ and Spectral Properties}

The Carré du Champ metric $g_{ij}$ defined in Section G is positive definite. This proof is self-contained and uses only spectral properties of the Laplacian, established in Sections D-F before metric emergence.

\noindent\textbf{Step 1: Carré du Champ Definition (Non-Circular)}

By Theorem \ref{thm:metricFromCarre} (Section G.2), the Carré du Champ operator is defined for the self-adjoint Laplacian $\mathcal{L}$ as:
\[
\Gamma(f, g) := \frac{1}{2}[\mathcal{L}(fg) - f \mathcal{L}(g) - g \mathcal{L}(f)].
\]

For eigenfunctions $\psi_i, \psi_j$ of the Laplacian (Theorem \ref{thm:discreteSpectrum} from Section D):
\[
\mathcal{L} \psi_i = \lambda_i \psi_i, \quad \mathcal{L} \psi_j = \lambda_j \psi_j,
\]
the Carré du Champ operator yields:
\[
\Gamma(\psi_i, \psi_j) = \nabla \psi_i \cdot \nabla \psi_j + \frac{\lambda_i + \lambda_j}{2} \psi_i \psi_j.
\]

The metric tensor is constructed from this operator via (Section G.3):
\[
g_{ij} := \sum_{k=1}^\infty \frac{1}{\lambda_k} \Gamma(\psi_i, \psi_j) [\psi_k].
\]

\noindent\textbf{Step 2: Positive Definiteness of Carré du Champ}

For any nonzero smooth function $f$, the Carré du Champ operator satisfies the fundamental property (Bakry-Émery theory):
\[
\Gamma(f, f) = \frac{1}{2}\mathcal{L}(f^2) - f \mathcal{L}(f) = |\nabla f|_{\mathrm{det}}^2 \geq 0,
\]
where $|\nabla f|_{\mathrm{det}}$ is the weak derivative of $f$ in the metric measure space. This is positive wherever $f$ is not constant (Theorem \ref{thm:carrePositivity}).

\noindent\textbf{Step 3: Positive Definiteness of the Metric}

The metric $g_{ij}$ is positive definite as a bilinear form. For any nonzero tangent vector $v = v^i \partial_i$ at a point $x \in X$:

\begin{enumerate}

\item The metric contraction is:
\[
g_{ij}(x) v^i v^j = \sum_{i,j} v^i v^j \sum_{k=1}^\infty \frac{1}{\lambda_k} \Gamma(\psi_i, \psi_j)[\psi_k(x)].
\]

\item Substituting the Carré du Champ formula:
\[
g_{ij} v^i v^j = \sum_k \frac{1}{\lambda_k} \Gamma(v, v)[\psi_k(x)],
\]
where $v = v^i \psi_i$ (the tangent vector expressed in the eigenfunction basis).

\item By Step 2, $\Gamma(v, v) \geq 0$ everywhere, with $\Gamma(v, v) > 0$ wherever $v$ is not constant.

\item Since the eigenfunctions $\{\psi_k\}$ form a complete orthonormal basis (Theorem \ref{thm:discreteSpectrum}), and the weights $1/\lambda_k$ are strictly positive:
\[
g_{ij} v^i v^j = \sum_k \frac{1}{\lambda_k} \Gamma(v, v)[\psi_k(x)] > 0 \quad \text{for all } v \neq 0.
\]

\end{enumerate}

This establishes positive definiteness of the metric tensor.

\noindent\textbf{Step 4: Independence from Temporal Structure}

The positivity of $g_{ij}$ is a property of the Carré du Champ operator alone, derived from:
\begin{itemize}
\item The self-adjoint Laplacian's discrete spectrum (Section D).
\item Heat kernel bounds (Section E).
\item The Carré du Champ definition (Section G.2).
\end{itemize}

It does NOT depend on:
\begin{itemize}
\item The temporal coordinate structure (developed in Sections I-J).
\item Lapse positivity (developed in Section K).
\item Lorentzian signature (developed in Section I).
\end{itemize}

Therefore, metric positivity is established \emph{independently and before} the development of temporal and causal structure, resolving any apparent circular dependence.

\noindent\textbf{Conclusion}

The Carré du Champ metric $g_{ij}$ is positive definite on the emergent manifold, regardless of the subsequent temporal foliation or signature choice. This establishes the Riemannian structure of spatial geometry as a self-contained consequence of spectral theory and the Bregman divergence structure (Axiom II), without forward references to later sections.
