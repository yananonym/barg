% proofThmYangMillsFieldOperators.tex
% Proof content


\begin{proof}

The proof establishes rigorous field operators for Yang-Mills theory through coherent state expansion and mollification.

\textbf{Step 1: Eigenfunction Expansion}

By Theorem \ref{thm:eigenfunctionRegularity}, the Dirichlet Laplacian $\Delta$ on $X$ has discrete spectrum with Hölder continuous eigenfunctions $\{\phi_n\}_{n=1}^\infty$ and eigenvalues $0 = \lambda_1 < \lambda_2 \leq \lambda_3 \leq \cdots \to \infty$. Expand the gauge field as:
\begin{equation}
\mathcal{A}(x) = \sum_{n=1}^\infty c_n(t) \phi_n(x),
\end{equation}
where $c_n(t)$ are time-dependent expansion coefficients.

\textbf{Step 2: Coherent State Construction}

For each mode $n$, define coherent states via the heat kernel regularization:
\begin{equation}
|\alpha_n(t)\rangle := \exp\left(-\frac{\lambda_n}{2}|\alpha_n|^2\right) \sum_{k=0}^\infty \frac{(\sqrt{\lambda_n} \alpha_n)^k}{\sqrt{k!}} |k_n\rangle,
\end{equation}
where $|k_n\rangle$ are Fock basis states for the $n$-th mode. By Lemma \ref{lem:mollifierCoherentStates}, these coherent states form an overcomplete basis of $\mathcal{H}_{YM}$ and satisfy the resolution of identity:
\begin{equation}
\int \frac{d\mu(\alpha_n)}{\pi} |\alpha_n(t)\rangle \langle \alpha_n(t)| = \mathbb{1}.
\end{equation}

\textbf{Step 3: Field Operator Definition via Mollification}

Define the mollified gauge field operator by:
\begin{equation}
\widehat{\mathcal{A}}_\mu^{\epsilon}(x, t) := \sum_{n=1}^\infty \rho_\epsilon(\lambda_n) \left(a_n e^{-i\omega_n t} + a_n^\dagger e^{i\omega_n t}\right) \phi_n(x),
\end{equation}
where:
\begin{itemize}
\item $\omega_n = \sqrt{\lambda_n}$ is the mode frequency
\item $\rho_\epsilon(\lambda_n) = \exp(-\epsilon \lambda_n / 2)$ is a ultraviolet cutoff
\item $a_n, a_n^\dagger$ are annihilation/creation operators satisfying $[a_n, a_m^\dagger] = \delta_{nm}$
\end{itemize}

The cutoff ensures $\widehat{\mathcal{A}}_\mu^\epsilon(x)$ is well-defined and tempered. The limit $\epsilon \to 0^+$ can be taken in correlation functions (Theorem \ref{thm:perturbationStability}).

\textbf{Step 4: Tempered Distribution Property}

For any test function $f \in \mathcal{S}(X \times \mathbb{R})$ (Schwartz class), the smeared field operator is:
\begin{equation}
\widehat{\mathcal{A}}_\mu(f) := \int_X \widehat{\mathcal{A}}_\mu(x, t) f(x, t) \, dt \, d\mu(x).
\end{equation}

For any state $|\psi\rangle \in \mathcal{H}_{YM}$, the functional $f \mapsto \langle \psi | \widehat{\mathcal{A}}_\mu(f) | \psi \rangle$ is a tempered distribution because:
\begin{equation}
|\langle \psi | \widehat{\mathcal{A}}_\mu(f) | \psi \rangle| \leq C \sum_{n=1}^\infty \rho_\epsilon(\lambda_n) \|f_n\|_{\mathcal{S}} \|\psi\|_{\mathcal{H}_{YM}}^2,
\end{equation}
where $f_n$ are expansion coefficients of $f$ in the eigenfunction basis, and the sum converges by the heat kernel bounds (Theorem \ref{thm:heatKernelBounds}).

\textbf{Step 5: Canonical Commutation Relations}

The canonical commutation relations follow from the bosonic algebra:
\begin{equation}
[\widehat{\mathcal{A}}_\mu(x), \widehat{\mathcal{A}}_\nu(y)] = 0, \quad [\hat{\pi}^\mu(x), \hat{\pi}^\nu(y)] = 0,
\end{equation}
where $\hat{\pi}^\mu = \partial_t \widehat{\mathcal{A}}^\mu - \nabla \hat{\Phi}$ is the canonically conjugate momentum. These are verified directly from the commutation relations of $a_n, a_n^\dagger$:
\begin{equation}
[a_n, a_m] = 0, \quad [a_n^\dagger, a_m^\dagger] = 0, \quad [a_n, a_m^\dagger] = \delta_{nm}.
\end{equation}

Locality (commutativity at spacelike separation) is ensured by the vanishing of the commutator for spacelike-separated test functions, by the spectral condition (Theorem \ref{thm:hadamardCondition}).

\textbf{Step 6: Cyclicity of Fock Space}

The Hilbert space $\mathcal{H}_{YM}$ is spanned by polynomials of creation operators applied to the vacuum:
\begin{equation}
\mathcal{H}_{YM} = \text{span}\left\{a_{n_1}^\dagger a_{n_2}^\dagger \cdots a_{n_k}^\dagger |0\rangle : k \in \mathbb{N}, n_1, \ldots, n_k \in \mathbb{N}\right\}.
\end{equation}

This follows from the Fock construction and completeness of the eigenfunction basis of $\Delta$.

\end{proof}
