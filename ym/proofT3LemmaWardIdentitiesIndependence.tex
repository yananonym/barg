% proofLemWardIdentitiesIndependence.tex
% Proof content


\begin{lemma}[Linear Independence of Ward Identity Constraints]
\label{lem:wardIdentitiesIndependence}

The three Ward identity constraints:
\begin{equation}
\mathcal{W}_a[\beta(g)] = 0, \quad a = 1, 2, 3,
\end{equation}
constitute automatic consequences of the fixed-point equations $\beta(g) = 0$. 

Rather, Ward identities are \textbf{emergent consistency conditions} that 
arise from gauge invariance of the path integral measure, independent of 
the divergence structure that generates the beta functions.

\begin{proof}

\textit{Step 1: Construct Beta Function from Divergence Structure Alone.}

The divergence-first framework generates beta functions from the renormalization 
of the Bregman divergence (Section X, Theorem \ref{thm:wardIdentitiesAllOrders}). 
These beta functions encode the RG flow on coupling space without explicit 
reference to gauge invariance. 

For a representative example, consider SU(3) Yang-Mills. The one-loop beta 
function derived purely from divergence geometry (without imposing gauge-invariance 
constraints explicitly) yields:
\begin{equation}
\beta_3^{(\mathrm{geom})} = -11 g_3^3 / (4\pi)^2 + O(g^5).
\end{equation}

This has a fixed point at $\beta_3^{(\mathrm{geom})}(g^*_3) = 0$ for some $g^*_3$.

\textit{Step 2: Check Ward Identity Satisfaction at This Fixed Point.}

However, the Ward identity from conservation of the gluon coupling under 
electroweak-gluon mixing (detailed in Theorem \ref{thm:wardIdentitiesAllOrders}) 
imposes a global constraint:
\begin{equation}
\beta_1(g^*) \cdot \kappa_1 + \beta_2(g^*) \cdot \kappa_2 + \beta_3(g^*) \cdot \kappa_3 = 0,
\end{equation}
where $\kappa_i$ are Dynkin indices of the respective gauge groups (depending 
on the gauge structure, these may be $2$ for U(1), $6$ for SU(2), $11$ for SU(3)).

\textit{Step 3: Demonstrate Non-Automatically Satisfied Condition.}

In general, a fixed point $g^* = (g^*_1, g^*_2, g^*_3)$ satisfying 
$\beta_i(g^*) = 0$ for $i = 1,2,3$ will not automatically satisfy the Ward 
identity constraint:
\begin{equation}
\mathcal{W}[\beta(g^*)] := \beta_1(g^*) \kappa_1 + \beta_2(g^*) \kappa_2 + \beta_3(g^*) \kappa_3 \neq 0
\end{equation}
unless the weights $\kappa_i$ and the fixed-point structure conspire to make 
the sum vanish. This is a non-trivial additional constraint.

\textit{Step 4: Conclude Independence.}

The three fixed-point equations $\beta_i(g^*) = 0$ form a system in three 
unknowns $g_1, g_2, g_3$, generically with isolated solutions. The Ward 
constraint $\mathcal{W}[\beta(g)] = 0$ is an additional linear (or polynomial) 
equation in the couplings. Only special combinations of the three couplings 
will simultaneously satisfy both $\beta(g) = 0$ and $\mathcal{W}[\beta(g)] = 0$.

Thus, the Ward identities are logically \emph{independent} restrictions beyond 
the divergence structure, and they further restrict the locus of fixed points 
from a generic zero-dimensional solution set to a  empty or discrete 
set depending on the gauge group and representation content.

\end{proof}

\end{lemma}
