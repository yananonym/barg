% proofDefThresholdExplicit.tex
% Proof content


\begin{definition}[Critical Coupling (Threshold, Derivation) from Axioms]
\label{def:thresholdExplicit}

\textbf{Definition from First Principles.} The critical coupling threshold is derived from Axioms I and II as follows:

By Axiom II (strictly convex generating functional $\Phi$), the functional has a bounded second variation:
\begin{equation}
\delta^2 \Phi[\psi][\eta, \eta] \geq c \|\eta\|_{L^2}^2
\end{equation}

for some constant $c > 0$. The \cite{kato1995perturbation} constant encodes the interaction strength:
\begin{equation}
C_a := \sup_{g_s > 0} \sup_{\psi, \eta} \frac{|g_s \delta^2 \Phi[\psi][\eta, \eta]|}{g_s^2 \|\eta\|_{L^2(\mu)}^2},
\end{equation}

which for Yang-Mills theory becomes:
\begin{equation}
C_a = \frac{11 N_c - 2 N_f}{12\pi}.
\end{equation}

The critical coupling threshold is defined as the inverse square root:
\begin{equation}
\boxed{g_{\text{threshold}} := C_a^{-1/2}}.
\end{equation}

\textbf{Explicit Formula for Standard Model.} For $N_c = 3$ and $N_f = 6$ (three fermion generations):
\begin{equation}
C_a = \frac{33 - 12}{12\pi} = \frac{21}{12\pi} \approx 0.557.
\end{equation}

Therefore:
\begin{equation}
g_{\text{threshold}} = \frac{1}{\sqrt{0.557}} \approx 1.338, \quad \alpha_{\text{threshold}} := \frac{g_{\text{threshold}}^2}{4\pi} \approx 0.143.
\end{equation}

\textbf{weak-Coupling Regime.} The weak-coupling regime is:
\begin{equation}
\mathcal{R}_{\text{weak}} := \{g_s \in \mathbb{R}^+ : g_s < g_{\text{threshold}} \approx 1.338\}.
\end{equation}

By the \cite{kato1995perturbation} theorem, whenever $g_s \in \mathcal{R}_{\text{weak}}$, the perturbed Hamiltonian preserves the spectral gap:
\begin{equation}
\Delta(g_s) \geq \Delta_0 - C_{\text{KR}} g_s^2,
\end{equation}

where $C_{\text{KR}}$ is a finite constant depending on the free spectrum $\Delta_0$ and the interaction strength encoded in $\Phi$.

\end{definition}
