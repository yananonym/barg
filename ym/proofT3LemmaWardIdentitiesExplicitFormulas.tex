% proofLemWardIdentitiesExplicitFormulas.tex
% Proof content


\begin{lemma}[Explicit Formulas for Ward Identity Constraints]
\label{lem:wardIdentitiesExplicitFormulas}

The three independent Ward identity constraints $\mathcal{W}_1, \mathcal{W}_2, \mathcal{W}_3$ on the beta functions $\beta(g) = (\beta_{g_1}, \beta_{g_2}, \beta_{g_3}, \beta_{y_t}, \beta_{y_b}, \beta_{y_\tau}, \beta_\lambda, \beta_{G_N}, \beta_\Lambda)$ are explicitly:

\begin{enumerate}

\item[\textbf{Ward Identity 1 (Diffeomorphism Invariance):}]
\begin{equation}
\mathcal{W}_1[\beta] := \beta_\Lambda + 4 \beta_{G_N} = 0.
\end{equation}

This encodes the Wess-Zumino consistency condition for the stress-energy tensor under coordinate transformations. Physically, it asserts that the cosmological constant and Newton constant renormalize in a coordinated way to preserve general covariance.

\item[\textbf{Ward Identity 2 (Electroweak Gauge Invariance):}]
\begin{equation}
\mathcal{W}_2[\beta] := g_1 \beta_{g_2} - g_2 \beta_{g_1} = C_{\mathrm{anom}}^{(EW)}(y_t, y_b, y_\tau, g_1, g_2, g_3),
\end{equation}

where $C_{\mathrm{anom}}^{(EW)}$ is the anomaly contribution from the fermion sector to electroweak running. This constraint ensures that the weak mixing angle $\sin^2 \theta_W = g_1^2/(g_1^2 + g_2^2)$ renormalizes consistently with the hypercharge and weak couplings.

At the fixed point $g = g^*$ where $\beta = 0$, this reduces to the anomaly cancellation requirement:
\begin{equation}
C_{\mathrm{anom}}^{(EW)}(g^*) = 0.
\end{equation}

\item[\textbf{Ward Identity 3 (Strong Gauge Invariance):}]
\begin{equation}
\mathcal{W}_3[\beta] := \beta_{g_3} - f_s(g_1, g_2, g_3, y_t, y_b) \cdot g_3 = C_{\mathrm{anom}}^{(strong)}(y_t, y_b, y_\tau, \text{matter content}),
\end{equation}

where $f_s$ is the strong coupling function arising from one-loop running and $C_{\mathrm{anom}}^{(strong)}$ is the anomaly contribution to strong gauge running. This constraint ensures that the strong coupling renormalizes consistently with matter content and gravitational coupling.

At the fixed point, anomaly cancellation requires:
\begin{equation}
C_{\mathrm{anom}}^{(strong)}(g^*) = 0.
\end{equation}

\end{enumerate}

\begin{proof}

Each Ward identity arises from Slavnov-Taylor consistency of the quantum effective action $\Gamma[g_i; \mu]$ at scale $\mu$ under gauge transformations.

\textbf{Ward Identity 1 (Diffeomorphism):} Under infinitesimal coordinate transformations $x^\mu \to x^\mu + \xi^\mu(x)$, the metric and all matter fields transform as tensors. The effective action obeys:

\begin{equation}
\delta_\xi \Gamma = \int d^4x \, \sqrt{g} \, \frac{\delta \Gamma}{\delta g_{\mu\nu}} \delta g_{\mu\nu} + \text{matter terms}.
\end{equation}

For diffeomorphism invariance, the variation must vanish (Wess-Zumino condition). At the level of the effective action, this translates to:

\begin{equation}
\beta_\Lambda(\mu) + 4 \beta_{G_N}(\mu) = 0 + O(\hbar^2),
\end{equation}

where the $O(\hbar^2)$ terms vanish at the one-loop level. This is the trace consistency condition.

\textbf{Ward Identity 2 (Electroweak):} Under infinitesimal $U(1)_Y \times SU(2)_L$ gauge transformations, the fermions and Higgs field transform under their respective representations. The Noether current conservation law in the quantum theory yields:

\begin{equation}
\partial_\mu \langle J_Y^\mu(x) \mathcal{O}(y) \rangle = \text{anomaly terms} \quad \text{(for } U(1)_Y\text{)},
\end{equation}

In RG language, this becomes a constraint on the running couplings:

\begin{equation}
g_1 \beta_{g_2} - g_2 \beta_{g_1} = (\text{anomaly from fermion loops}).
\end{equation}

\textbf{Ward Identity 3 (Strong):} By the same logic for $SU(3)_c$ gauge transformations, the constraint is:

\begin{equation}
\beta_{g_3} = \text{(function of couplings)} \pm (\text{anomaly terms}).
\end{equation}

\textbf{Linearity and Independence:} Each Ward identity is linear in the beta functions (equations 1 and 2 are manifestly linear; equation 3 is linear with an inhomogeneous anomaly term). By Lemma \ref{lem:wardIdentitiesIndependence}, these three are linearly independent functionals of $\beta$.

\textbf{Codimension in Coupling Space:} Each Ward identity is one scalar equation in the 9-dimensional coupling space $\mathcal{G}$. Since the three constraints are linearly independent, they define a codimension-3 constraint surface $\mathcal{S}_6 = \{g : \mathcal{W}_1 = 0, \mathcal{W}_2 = 0, \mathcal{W}_3 = 0\}$.

\qed

\end{proof}

\end{lemma}

\begin{remark}[Physical Interpretation of Ward Identities]
\label{rem:wardIdentitiesPhysical}

The three Ward identities enforce that quantum corrections preserve the classical gauge symmetries. Without these constraints, the RG flow would drive the theory away from gauge invariance at high energies. The existence of an asymptotically safe fixed point satisfying all three Ward identities simultaneously is therefore a highly non-trivial achievement: it confirms that gauge symmetry is not broken by quantum effects in the UV limit.

\end{remark}
