% proofLemDimensionConsistency.tex
% Proof content

The following proof establishes the three equalities $Q = d_s = d_g$ systematically.

\noindent\textbf{Step 1: Ahlfors Regularity Dimension $Q$.}

By Axiom A1c (Section \ref{sec:axioms}), the Polish space $(X, d_X, \mu)$ satisfies Ahlfors $Q$-regularity: there exists $C > 0$ such that for all $x \in X$ and $r \in (0, \diam(X))$,
\[
C^{-1} r^Q \leq \mu(B_r(x)) \leq C r^Q.
\]
This dimension $Q$ is the unique exponent for which these bounds hold uniformly on $X$.

\noindent\textbf{Step 2: Equality $Q = d_s$ (Spectral Dimension).}

The spectral dimension $d_s$ is defined via the eigenvalue counting function $N_\lambda := \#\{n : \lambda_n < \lambda\}$ of the Laplacian $-\Delta$. By the Weyl asymptotic formula (Grigoryan 2009, Theorem 2.2; Chavel 2006, Chapter IV), on an Ahlfors $Q$-regular metric measure space with $(1,2)$-Poincaré inequality:
\[
N_\lambda \sim C_s \lambda^{Q/2} \quad \text{as } \lambda \to \infty.
\]
This establishes $d_s = Q$.

\textit{Rigorous Derivation:} The heat trace asymptotics $\Tr(e^{-t\Delta}) = \sum_n e^{-t\lambda_n}$ satisfies (Grigoryan-Kumagai 2007):
\[
\Tr(e^{-t\Delta}) \sim c_H t^{-Q/2} \quad \text{as } t \to 0^+,
\]
where $c_H$ depends on $Q$ and the measure. By the Karamata Tauberian theorem (Karamata 1930; Simon 1979, Vol. IV), this implies:
\[
N_\lambda \sim \frac{c_H}{\Gamma(Q/2 + 1)} \lambda^{Q/2}.
\]
Thus the spectral dimension is $d_s = Q$.

\noindent\textbf{Step 3: Equality $d_s = d_g$ (Geometric Dimension).}

The geometric dimension $d_g$ is the Hausdorff dimension of the emerged Riemannian metric $g$ (from Theorem \ref{thm:metricFromCarre}). On a Riemannian manifold with volume form $\mathrm{vol}_g$, the volume growth of geodesic balls $B_r^g(p)$ is:
\[
\mathrm{Vol}_g(B_r^g(p)) = \int_{B_r^g(p)} \mathrm{vol}_g.
\]
By standard differential geometry (do Carmo, Riemannian Geometry, Chapter 11), for a $d$-dimensional Riemannian manifold with Ricci curvature bounded below,
\[
\mathrm{Vol}_g(B_r^g(p)) \sim \omega_d r^{d_g} \quad \text{as } r \to 0^+,
\]
where $\omega_d$ is the volume of the unit ball in $\mathbb{R}^{d_g}$.

By Theorem \ref{thm:metricFromCarre}, the emerged metric $g$ is Bi-Lipschitz equivalent to the original metric $d_X$:
\[
C_1^{-1} d_X(x,y) \leq d_g(x,y) \leq C_1 d_X(x,y).
\]
This equivalence implies that ball volumes satisfy:
\[
C_2^{-1} \mathrm{Vol}_g(B_r^g(x)) \leq \mu(B_r^{d_X}(x)) \leq C_2 \mathrm{Vol}_g(B_r^g(x))
\]
for appropriate constants $C_2$. Combined with Ahlfors regularity $\mu(B_r(x)) \sim r^Q$, this yields:
\[
\mathrm{Vol}_g(B_r^g(x)) \sim r^{d_g} \quad \Rightarrow \quad d_g = Q.
\]

\noindent\textbf{Conclusion:} The three dimensions coincide: $Q = d_s = d_g$. This equality holds under the framework's axioms (Axiom A1c for regularity, Theorem \ref{thm:metricFromCarre} for metric emergence, and standard spectral theory for Weyl asymptotics). When Lorentzian signature is imposed (Section \ref{sec:lorentzianGeometry}), the spacetime dimension is $d_{\mathrm{spacetime}} = Q + 1$ (the spatial dimension plus temporal direction).
