% proofLemLapseStrictPositive.tex
% Proof content

\begin{proof}
The following derivation establishes $N(x) \geq N_{\min} > 0$ uniformly on $X$.

\textbf{Step 1: Continuity of Lapse.}
By Theorem~\ref{thm:lapsePositivityFromDivergence}, the lapse function is:
\[
N(x) = |T[\psi_0](x)| = \left|\frac{\delta \mathcal{A}}{\delta \psi}[\psi_0](x)\right|.
\]
Since $\psi_0 \in C^{0,\alpha}(X)$ (Lemma~\ref{lem:vacuumProperties}) and 
$V''' \in C^0(\mathbb{R}_+)$ (Axiom 2, item V1), the functional derivative 
yields $N \in C^{0,\beta}(X)$ for some $\beta > 0$ by composition of continuous maps.

\textbf{Step 2: Positivity from Non-Degeneracy.}
By Lemma~\ref{lem:asymmetryProperties}(4), the asymmetry functional is 
non-degenerate: the operator $T: \psi \mapsto \delta\mathcal{A}/\delta\psi[\cdot]$ 
is an isomorphism on its domain.

The lapse function $N(x)$ vanishes only at critical points of the functional $\psi \mapsto \mathcal{A}[\psi, \psi_0]$. By strict convexity of $V$ (Axiom 2, V2), the vacuum $\psi_0$ is the unique minimizer of the action, hence the only critical point.

Moreover, the third-derivative term dominates: 
\[
T[\psi_0](x) = V'''(|\psi_0(x)|^2) |\psi_0(x)|^4 + O(|\psi_0|^6)
\]
is strictly positive where $\psi_0$ is non-zero (which is everywhere since $\supp(\psi_0) = X$ by full-measure vacuum condition).

By the Morse-Sard theorem on $C^{1,\beta}$ functions on metric measure spaces 
(Bates 1993), the critical set $\{x : N(x) = 0\}$ has measure zero. But 
$N$ is continuous on compact $X$, so this set is closed. A closed measure-zero set in a 
compact connected space with full-support measure must be empty (Lemma~\ref{lem:domainDensity}).

Therefore $N(x) > 0$ everywhere on $X$.

\textbf{Step 3: Uniform Lower Bound via Compactness.}
Since $N: X \to (0, \infty)$ is continuous on compact $X$ and everywhere positive:
\[
N_{\min} := \inf_{x \in X} N(x) = \min_{x \in X} N(x) > 0
\]
by the extreme value theorem (continuous function on compact set attains its infimum).

\textbf{Step 4: Explicit Dependence on Axiom Data.}
The constant $N_{\min}$ depends on:
\begin{itemize}
\item Axiom constants: $C_A, C_P, Q, \lambda_0, \Lambda_0$
\item Vacuum configuration: $\|\psi_0\|_{C^{0,\alpha}(X)}$
\item Potential third derivative: $\|V'''\|_{C^0([0, \|\psi_0\|_\infty^2])}$
\item Compactness diameter: $\diam(X)$
\end{itemize}
All are determined by the axiom data and finitude of the configuration space. \qed
\end{proof}