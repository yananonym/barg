% proofLLemmaDimensionalIntegralityFromManifoldEmbedding.tex
% Proof of dimensional integrality constraint

\begin{lemma}[Dimensional Integrality from Manifold Embedding]
\label{lem:dimensionalIntegrality}

Let $X$ be a Polish space satisfying Axiom I with Ahlfors regularity dimension $Q \in (2, \infty)$. 
If $X$ admits:
\begin{enumerate}
  \item[(i)] An embedded smooth Riemannian metric via the Carré du Champ construction (Theorem \ref{thm:metricFromCarre}), and
  \item[(ii)] A spectral embedding $\Phi: X \to \mathbb{R}^N$ into a smooth submanifold $M^d$ 
       (Theorem \ref{thm:spectralEmbedding}),
\end{enumerate}
then the Ahlfors dimension must satisfy $d = \dim(M) = Q \in \mathbb{Z}$.

\begin{proof}

The following derivation establishes this result in three steps:

\textbf{Step 1: Manifold Dimension is Integer-Valued by Definition}

By differential geometry (Whitney embedding theorem and manifold theory), any smooth $d$-dimensional Riemannian manifold embedded in $\mathbb{R}^N$ has topological dimension $d \in \mathbb{Z}_{\geq 1}$. The manifold dimension is defined as the dimension of the tangent space at each point, which is an integer by the definition of a manifold.

More precisely: if $M \subset \mathbb{R}^N$ is a smooth submanifold, then at each point $p \in M$, the tangent space $T_p M$ is a finite-dimensional vector space over $\mathbb{R}$. The dimension $\dim(T_p M)$ is a positive integer $d$.

\textbf{Step 2: The Spectral Embedding Produces a Smooth Submanifold}

By Theorem \ref{thm:spectralEmbedding}, the eigenfunction coordinates $(\lambda_1, \lambda_2, \ldots, \lambda_N)$ of the Laplacian on $X$ define a continuous embedding:
\begin{equation}
\Phi: X \to \mathbb{R}^N, \quad \Phi(x) = (\lambda_1(x), \lambda_2(x), \ldots, \lambda_N(x)).
\end{equation}

By Theorem \ref{thm:eigenfunctionRegularity}, the eigenfunctions $\lambda_i \in C^{0,\alpha}(X)$ with $\alpha = 1 - Q/4 > 0$ (for $Q < 4$). The Carré du Champ construction (Theorem \ref{thm:metricFromCarre}) shows that:
\begin{equation}
g_{ij} = \Gamma(\lambda_i, \lambda_j)
\end{equation}
defines a Riemannian metric on $X$. This metric is smooth (Hölder continuous with $\beta > 0$), so the pullback metric on the image $M = \Phi(X) \subset \mathbb{R}^N$ is smooth.

By the implicit function theorem applied to the spectral constraints, $M$ is a smooth submanifold of $\mathbb{R}^N$.

\textbf{Step 3: Ahlfors Dimension Equals Manifold Dimension}

The Ahlfors dimension of $X$ is defined through the measure $\mu$:
\begin{equation}
\mu(B_r(x)) \sim r^Q.
\end{equation}

The spectral embedding $\Phi$ preserves the measure (up to a bounded factor) by Lemma \ref{lem:uniformMetricNondegeneracy}. Therefore, the measure on $M = \Phi(X)$ also satisfies Ahlfors regularity with the same exponent $Q$.

On a smooth $d$-dimensional Riemannian manifold $M$, the volume form satisfies:
\begin{equation}
\text{Vol}(B_r(p)) \sim r^d.
\end{equation}

By standard measure theory, the Ahlfors dimension equals the Hausdorff dimension, which on a smooth manifold equals the topological dimension:
\begin{equation}
Q = d = \dim(M).
\end{equation}

Since the manifold dimension is an integer, $Q \in \mathbb{Z}$.

\textbf{Conclusion:}

The Ahlfors dimension of a Polish space that admits a smooth Riemannian metric and spectral embedding must be an integer. Therefore, among the constraint values $Q \in (2, 4)$, only the integer values $Q \in \{3\}$ are possible when combined with the requirement $Q < 4$ for smooth metric emergence.

\qed

\end{proof}

\end{lemma}
