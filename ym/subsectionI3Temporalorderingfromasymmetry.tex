% Part of sectionITemporalCausality.tex
\subsection{Temporal Ordering from Asymmetry}
\label{subsec:temporalOrderingFromAsymmetry}

\begin{definition}[Temporal Vector Field - Domain Specification]
\label{def:temporalVectorFieldDomain}
The temporal vector field $T[\psi]$ is defined as the Frechet derivative of the asymmetry functional with respect to its first argument:
\begin{equation}
\langle T[\psi], h \rangle := D_\psi \mathcal{A}[\psi, \psi_0] \cdot h = \lim_{\epsilon \to 0} \frac{1}{\epsilon}[\mathcal{A}[\psi + \epsilon h, \psi_0] - \mathcal{A}[\psi, \psi_0]],
\end{equation}
where $\psi_0$ is the reference vacuum configuration from Definition \ref{def:vacuumState}.

\textbf{Domain of Frechet Differentiability:}

The functional $\mathcal{A}[\cdot, \psi_0]$ is Frechet differentiable on the domain:
\begin{equation}
\Dom(T) := \{\psi \in \mathcal{H} : D[\psi\|\psi_0] < \infty \text{ and } D[\psi_0\|\psi] < \infty \}.
\end{equation}

By Definition \ref{def:bregman} and Axiom \ref{ax:configSpace}, this domain is:
\begin{equation}
\Dom(T) = \left\{\psi \in L^2(X, \mu; \mathbb{C}^n) : V(|\psi|^2), V(|\psi_0|^2) \in L^1(X, \mu), \quad |V'(|\psi_0|^2)| |\psi|^2 \in L^1(X, \mu)\right\}.
\end{equation}

By polynomial growth of $V$ (condition V3 in Axiom \ref{ax:configSpace}), this domain is dense in $\mathcal{H} = L^2(X, \mu; \mathbb{C}^n)$ and contains $H^{1,2}(X) \otimes \mathbb{C}^n$ (for $Q < 4$).

\textbf{Frechet Derivative Explicit Formula:}

For $\psi \in \Dom(T)$:
\begin{equation}
\langle T[\psi], h \rangle = 2 \int_X V'(|\psi|^2) \text{Re}(\overline{\psi} \cdot h) d\mu - 2 \int_X V'(|\psi_0|^2) \text{Re}(\overline{\psi_0} \cdot h) d\mu.
\end{equation}

This is a bounded linear functional in $h \in \mathcal{Hölder inequality and polynomial growth, establishing Frechet differentiability. The vector field $T[\psi]$ is thus well-defined as an element of the dual space $\mathcal{H}^*$, which can be identified with $\mathcal{H}$ by Riesz representation.

\textbf{Continuity:} The map $\psi \mapsto T[\psi]$ is continuous in the $\mathcal{H}$-norm on its domain by continuity of the Bregman divergence.
\end{definition}

\begin{definition}[Path-Based Causal Ordering - Exact]
\label{def:pathCausalOrdering}
For configurations $\psi, \phi \in \Dom(T)$, define the causal relation $\phi \prec \psi$ (read: ``$\phi$ is in the causal past of $\psi$'') if there exists a continuous path $\gamma: [0,1] \to \Dom(T)$ with:

\begin{enumerate}[label=(\roman*)]
\item $\gamma(0) = \phi$ and $\gamma(1) = \psi$.
\item $\gamma$ is differentiable almost everywhere with respect to Lebesgue measure on $[0,1]$.
\item The \textbf{integrated asymmetry} is strictly positive:
\begin{equation}
\int_0^1 \langle T[\gamma(s)], \gamma'(s) \rangle ds > 0,
\end{equation}
where the inner product is taken in $\mathcal{H}$.
\end{enumerate}
\end{definition}

\begin{lemma}[Path-Independence in Infinite Dimensions]
\label{lem:pathIndependenceInfiniteDim}
The temporal potential $t[\psi] := \mathcal{A}[\psi, \psi_0]$ is path-independent in the sense that for any two paths $\gamma_1, \gamma_2: [0,1] \to \Dom(T)$ with $\gamma_1(0) = \gamma_2(0) = \phi$ and $\gamma_1(1) = \gamma_2(1) = \psi$:

\begin{equation}
\int_0^1 \langle T[\gamma_1(s)], \gamma_1'(s) \rangle ds = \int_0^1 \langle T[\gamma_2(s)], \gamma_2'(s) \rangle ds,
\end{equation}

and both equal $t[\psi] - t[\phi] = \mathcal{A}[\psi, \psi_0] - \mathcal{A}[\phi, \psi_0]$.

\begin{proof}
% proofLemPathIndependenceInfiniteDim.tex
% Proof content

The verify the three conditions for path-independence in infinite-dimensional Banach spaces (Lang, Differential and Riemannian Manifolds, Ch. III).

\textit{(i) Frechet Differentiability on Open Connected Domain.}

On $\Dom(T) = \{\psi : D[\psi\|\psi_0], D[\psi_0\|\psi] < \infty\}$, the functional $\mathcal{A}[\cdot, \psi_0]$ is Frechet differentiable by Definition \ref{def:temporalVectorFieldDomain} (polynomial growth of $V$ from Axiom \ref{ax:configSpace} condition V3 ensures dominated convergence applies).

The Frechet derivative is $D_\psi \mathcal{A}[\cdot, \psi_0] = T[\psi]$, which is linear and bounded in $h$.

Moreover, $\Dom(T)$ is an open subset of $\mathcal{H}$ (as the sublevel set of the lower-semicontinuous function $\max(D[\cdot\|\psi_0], D[\psi_0\|\cdot])$) and is connected (by convexity of sublevel sets of convex functions).

\textit{(ii) Simple Connectivity of Domain.}

The key observation is that $\Dom(T)$ is a convex subset of $\mathcal{H}$. To see this: the Bregman divergence $D[\psi \| \phi]$ is convex in its first argument $\psi$ (this follows from strict convexity of $V$, condition V2). Therefore:
\begin{equation}
\{\psi : D[\psi\|\psi_0] < \infty\}
\end{equation}
is a convex set (sublevel set of convex function).

Since any convex subset of a Banach space is simply connected, $\Dom(T)$ is simply connected. This eliminates topological obstructions to path-independence.

\textit{(iii) Fundamental Theorem of Calculus in Banach Spaces.}

For any $C^1$ path $\gamma: [0,1] \to \Dom(T)$ (i.e., differentiable with continuous derivative), the fundamental theorem of calculus for Banach spaces (Lang 1995, Dieudonne 1969) states:

\begin{equation}
f(\gamma(1)) - f(\gamma(0)) = \int_0^1 \langle D f[\gamma(s)], \gamma'(s) \rangle ds,
\end{equation}
where $f$ is Frechet differentiable and the inner product is taken between the Frechet derivative (in the dual space $\mathcal{H}^*$) and the tangent vector (in $\mathcal{H}$).

Applying this with $f = \mathcal{A}[\cdot, \psi_0]$:
\begin{equation}
\mathcal{A}[\gamma(1), \psi_0] - \mathcal{A}[\gamma(0), \psi_0] = \int_0^1 \langle T[\gamma(s)], \gamma'(s) \rangle ds,
\end{equation}
where $T[\gamma(s)] = D_\psi \mathcal{A}[\gamma(s), \psi_0]$ from Definition \ref{def:temporalVectorFieldDomain}.

\textit{(iv) Path-Independence Conclusion.}

Since the right-hand side depends only on the starting point $\gamma(0)$, the ending point $\gamma(1)$, and the value of $\mathcal{A}[\cdot, \psi_0]$ (not on the path itself), there is:

For any two paths $\gamma_1, \gamma_2$ connecting $\phi$ to $\psi$:
\begin{equation}
\int_0^1 \langle T[\gamma_1(s)], \gamma_1'(s) \rangle ds = \mathcal{A}[\psi, \psi_0] - \mathcal{A}[\phi, \psi_0] = \int_0^1 \langle T[\gamma_2(s)], \gamma_2'(s) \rangle ds.
\end{equation}

Therefore, the line integral $\int \langle T, d\gamma \rangle$ is independent of the path, and the temporal potential $t[\psi] := \mathcal{A}[\psi, \psi_0] - \mathcal{A}[\psi_0, \psi_0] = \mathcal{A}[\psi, \psi_0]$ (since $\mathcal{A}[\psi_0, \psi_0] = 0$) is a well-defined scalar function on $\Dom(T)$. \qed

\end{proof}
\end{lemma}

\begin{remark}[Why Infinite-Dimensional Path-Independence is Non-Trivial]
\label{rem:whyinfinitedimensionalpathindependenceisnontrivial}
In finite-dimensional Euclidean spaces, path-independence is automatic for any gradient field (since gradients integrate to potentials by the fundamental theorem). However, in infinite-dimensional Banach spaces like $\mathcal{H}$, one must verify:

\begin{enumerate}
\item Frechet differentiability (not just Gateaux differentiability)
\item Simple connectivity of the domain (topological obstruction)
\item Applicability of the fundamental theorem (requires sufficient regularity)
\end{enumerate}

All three conditions are satisfied here due to (i) polynomial growth of $V$, (ii) convexity of $\Dom(T)$, and (iii) Frechet differentiability of $\mathcal{A}[\cdot, \psi_0]$.
\end{remark}

