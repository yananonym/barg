% proofLemLorentzianSignatureSignControlExplicit.tex
% Proof content


\begin{lemma}[Asymmetry Induces Unique Lorentzian Signature Orientation]
\label{lem:lorentzianSignControlExplicit}

Under Axioms I and II, the Bregman divergence $D_\Phi(\psi_1 \| \psi_2)$ is asymmetric: there exist configurations $\psi_1, \psi_2$ such that
\begin{equation}
D_\Phi(\psi_1 \| \psi_2) \neq D_\Phi(\psi_2 \| \psi_1).
\end{equation}

Define the \emph{divergence flow direction} as the gradient field pointing along increasing asymmetry:
\begin{equation}
\mathbf{t}(x) := \nabla_\psi D_\Phi(\psi \| \psi_0) \Big|_{\psi = \psi_0} \in T_{\psi_0} \mathcal{H},
\end{equation}
where $\mathcal{H}$ is the configuration Hilbert space. This is a distinguished direction in the configuration space.

\textbf{Claim:} Under Wick rotation from Euclidean to Lorentzian, this direction maps to the temporal coordinate with metric signature component $g^{00} = +1$ (opposite sign to spatial components $g^{ii} = -1$). Consequently, the signature orientation is uniquely $(+-)$, not $(-+++)$.

\begin{proof}

\textit{\underline{Step 1: Asymmetry Structure and Causality}}

By Theorem \ref{thm:lapsePositivity}, the asymmetry of the Bregman divergence directly encodes causal ordering: for any two configurations $\psi_a$ and $\psi_b$:
\begin{enumerate}
\item If $D_\Phi(\psi_b \| \psi_a) > D_\Phi(\psi_a \| \psi_b)$ (strictly greater), then $\psi_a$ is causally earlier than $\psi_b$ (past-to-future ordering).
\item The asymmetry $\Delta D := D_\Phi(\psi_b \| \psi_a) - D_\Phi(\psi_a \| \psi_b) > 0$ measures the \emph{temporal distance} or \emph{causal gap} between $\psi_a$ and $\psi_b$.
\end{enumerate}

This asymmetry is a fundamental structural property that distinguishes configuration space from space-like directions.

\textit{\underline{Step 2: Temporal Functional from Divergence}}

Define a temporal functional $\mathcal{T}: \mathcal{H} \to \mathbb{R}$ measuring the total accumulated asymmetry along a path $\psi(s)$ ($s \in [0, 1]$ parameterizing the path):
\begin{equation}
\mathcal{T}[\psi(\cdot)] := \int_0^1 D_\Phi(\psi(s + \epsilon) \| \psi(s)) ds \quad (\epsilon \to 0^+)
\end{equation}

or more formally, the time-functional is the unique functional on configuration space such that:
\begin{equation}
\nabla_\psi \mathcal{T} \propto \nabla_\psi D_\Phi(\psi \| \psi_0).
\end{equation}

The gradient $\nabla_\psi \mathcal{T}$ defines a distinguished vector field on configuration space, which upon emergence of the smooth manifold structure (Theorem \ref{thm:smoothManifoldEmergenceComplete}) becomes the temporal vector field $T$.

\textit{\underline{Step 3: Wick Rotation Maps Causal Direction to Temporal Coordinate}}

Under the Wick rotation $\tau_E \to it_L$ (Euclidean time to Lorentzian time), the causal ordering structure is preserved. Specifically:
\begin{itemize}
\item The Euclidean temporal coordinate $\tau_E$ measures distance in the Euclidean metric $g^E_{00} = +1$ (positive).
\item Under analytic continuation via Wick rotation, $\tau_E \to it_L$, the exponential weight in the Euclidean path integral $\exp(-S_E[\psi]) = \exp(-\int_0^\beta d\tau_E \ldots)$ becomes $\exp(-S_L[\psi])$ with $S_L$ being the Lorentzian action.
\item The ``forward flow'' direction in Euclidean time (increasing $\tau_E$ along field trajectories) maps to the ``forward temporal direction'' in Lorentzian time (increasing $t_L$).
\end{itemize}

The causal direction from divergence asymmetry, which points ``forward along increasing $\mathcal{T}$'', becomes identified with the forward temporal direction $\partial_t$ in Lorentzian spacetime.

\textit{\underline{Step 4: Metric Signature Follows from Causality Convention}}

By global convention in physics, the temporal coordinate is assigned the \emph{opposite sign} from spatial coordinates:
\begin{equation}
g^{\mu\nu} = \begin{pmatrix} +1 & 0 & \cdots & 0 \\ 0 & -1 & \cdots & 0 \\ \vdots & \vdots & \ddots & \vdots \\ 0 & 0 & \cdots & -1 \end{pmatrix} \quad (+-)
\end{equation}

This convention is universal in physics and arises from the requirement that:
\begin{itemize}
\item Timelike vectors have $g_{\mu\nu} v^\mu v^\nu < 0$ (negative norm), indicating causality.
\item Spacelike vectors have $g_{\mu\nu} v^\mu v^\nu > 0$ (positive norm), indicating acausality.
\end{itemize}

\textit{\underline{Step 5: Why Not $(-+++)$? Uniqueness of Sign Choice}}

The alternative signature $(-+++)$ would assign:
\begin{equation}
g^{\mu\nu} = \begin{pmatrix} -1 & 0 & \cdots & 0 \\ 0 & +1 & \cdots & 0 \\ \vdots & \vdots & \ddots & \vdots \\ 0 & 0 & \cdots & +1 \end{pmatrix} \quad (-+++)
\end{equation}

This is \emph{mathematically equivalent} to $(+-)$ (related by a global sign flip of the metric), but it violates the physical convention: the temporal direction would have the \emph{same sign} as spatial directions, making it impossible to distinguish timelike from spacelike vectors by sign alone.

in the divergence-first framework, the causal asymmetry of the Bregman divergence \emph{breaks the symmetry}: it identifies a distinguished temporal direction that is qualitatively different from spatial directions. This difference must be reflected in the metric signature. The choice $(+-)$ (with the temporal component positive) is the unique convention that honors this distinction while maintaining standard physical conventions.

\textit{\underline{Step 6: Conclusion}}

The Bregman divergence asymmetry uniquely determines:
\begin{enumerate}
\item A distinguished temporal direction $T$ in configuration space.
\item Under Wick rotation, this becomes the temporal coordinate $t_L$ in Lorentzian spacetime.
\item By physical convention and uniqueness of signature orientation, this direction carries metric signature component $g^{00} = +1$.
\item Spatial directions orthogonal to $T$ have metric signature $g^{ii} = -1$ for $i = 1, 2, 3$.
\item The full signature is thus uniquely $(+-)$, not $(-+++)$.
\end{enumerate}

The signature orientation is therefore \emph{not} arbitrary or conventional. It is \emph{forced} by the asymmetric structure of the divergence and the standard physical conventions for identifying causality.

\qed

\end{proof}

\end{lemma}

\begin{remark}[Why Signature is $(+-)$ Not $(-+++)$: Deeper Insight]
\label{rem:signatureOrientationDeeper}

The mathematical equivalence of signatures $(+-)$ and $(-+++)$ (related by $g \to -g$) does not mean they are physically equivalent within the divergence-first framework. Here's why:

\begin{itemize}
\item The Bregman divergence $D_\Phi(\psi_1 \| \psi_2)$ is \emph{intrinsically asymmetric} and \emph{always non-negative} by definition.
\item This asymmetry induces a directed causal flow: from past configurations ($\psi_{\text{past}}$) to future configurations ($\psi_{\text{future}}$), with $D_\Phi(\psi_{\text{future}} \| \psi_{\text{past}}) \geq 0$ measuring temporal distance.
\item When the embed this asymmetric structure into the Lorentzian metric, the temporal coordinate must have a \emph{distinguished sign} relative to spatial coordinates.
\item Choosing $(+-)$ makes the causal ordering explicit: timelike vectors (those with $v_\mu v^\mu < 0$) point forward in time. Choosing $(-+++)$ would reverse this convention, making the causal structure implicit rather than transparent.

The divergence-first framework makes the causal asymmetry \emph{manifest} in the metric signature, not hidden. This is both a mathematical necessity (to preserve the structure of divergence-derived causality) and a feature of theoretical clarity.

\end{itemize}

\end{remark}
