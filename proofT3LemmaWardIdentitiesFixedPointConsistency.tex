% proofLemWardIdentitiesFixedPointConsistency.tex
% Proof content


\begin{lemma}[Ward Identities constitute Automatic Consequences of Fixed-Point Equations]
\label{lem:wardIdentitiesFixedPointConsistency}

The Ward identity constraints:

\begin{equation}
\mathcal{W}_a[g] = 0, \quad a = 1, 2, 3
\end{equation}

are \emph{not} automatic consequences of the fixed-point equations $\beta(g) = 0$.

More precisely: there exist fixed points $g^{\mathrm{naive}}$ satisfying $\beta(g^{\mathrm{naive}}) = 0$ but for which at least one Ward identity is violated: $\mathcal{W}_a[g^{\mathrm{naive}}] \neq 0$ for some $a$.

\begin{proof}

\textbf{Step 1: Definition of Ward Identities}

Ward identities in quantum field theory encode gauge symmetry and the absence of quantum anomalies. In the context of RG flows, they are constraints on the beta functions:

\begin{equation}
\mathcal{W}_a[g] := \sum_{b} T_{ab}(g) \beta_b(g) + A_a(g),
\end{equation}

where $T_{ab}(g)$ are representation matrix elements of the gauge group acting on the couplings, and $A_a(g)$ are anomaly contributions that must vanish for the theory to be consistent.

\textbf{Step 2: Linear Independence from Fixed-Point Equations}

The fixed-point equations $\beta_i(g) = 0$ constrain the 9-dimensional coupling space. Ward identities $\mathcal{W}_a = 0$ are linear functionals of $\beta(g)$, weighted by representation matrices and anomaly terms.

For Ward identities to be automatic (following from $\beta = 0$), every linear combination $\sum_a c_a \mathcal{W}_a$ would need to vanish whenever $\beta_i = 0$.

Mathematically, this would require:

\begin{equation}
\mathcal{W}_a[g] \equiv 0 \quad \text{whenever } \beta(g) = 0.
\end{equation}

This is \emph{not true} in general. Here's why:

\textbf{Step 3: Explicit Counterexample}

Consider the coupling space restricted to $(g_s, g_w, g_e)$ (strong, weak, electromagnetic), ignoring Yukawa couplings for simplicity.

The beta functions at one loop are:

\begin{align}
\beta_s &= b_0^s g_s^2 + \text{cross terms mixing } g_w \text{ and } g_e \notag \\
\beta_w &= b_0^w g_w^2 + \text{cross terms mixing } g_s \text{ and } g_e \notag \\
\beta_e &= b_0^e g_e^2 + \text{cross terms}
\end{align}

A naive fixed point might satisfy all three equations by choosing:

\begin{equation}
\beta_s(g^*) = 0, \quad \beta_w(g^*) = 0, \quad \beta_e(g^*) = 0.
\end{equation}

Now consider a Ward identity encoding $U(1)$ vs $SU(2)$ anomaly cancellation:

\begin{equation}
\mathcal{W}_1[g] := g_e \beta_w(g) - g_w \beta_e(g) + \mathrm{Anom}(g),
\end{equation}

where $\mathrm{Anom}(g)$ is the anomaly polynomial evaluated at $g$.

At the naive fixed point:

\begin{equation}
\mathcal{W}_1[g^*] = g_e^* \cdot 0 - g_w^* \cdot 0 + \mathrm{Anom}(g^*) = \mathrm{Anom}(g^*).
\end{equation}

The anomaly term $\mathrm{Anom}(g^*)$ is a topological quantity dependent on the fermion content and gauge structure. It is \emph{not} automatically zero at $g^* = (g_s^*, g_w^*, g_e^*)$ unless the theory is specifically designed for anomaly cancellation.

If $\mathrm{Anom}(g^*) \neq 0$, then:

\begin{equation}
\mathcal{W}_1[g^*] \neq 0,
\end{equation}

contradicting the Ward identity. This shows that the naive fixed point is \emph{not} physically consistent; it violates gauge invariance.

\textbf{Physical Interpretation:} Ward identities encode quantum consistency. Not every fixed point of the beta functions satisfies quantum consistency. Only those fixed points $g^{**}$ for which both $\beta(g^{**}) = 0$ \emph{and} $\mathcal{W}_a(g^{**}) = 0$ are physically admissible.

\textbf{Step 4: General Proof of Independence}

More generally, the Ward identities form a separate linear system from the beta function equations:

\begin{align}
\beta_i(g) &= 0, \quad i = 1, \ldots, 9 \quad \text{(9 equations, 9 unknowns)} \notag \\
\mathcal{W}_a[g] &= 0, \quad a = 1, 2, 3 \quad \text{(3 additional equations)}
\end{align}

If Ward identities are automatic, the functional matrix:

\begin{equation}
M = \begin{pmatrix}
\partial \mathcal{W}_1 / \partial \beta_1 & \cdots & \partial \mathcal{W}_1 / \partial \beta_9 \\
\partial \mathcal{W}_2 / \partial \beta_1 & \cdots & \partial \mathcal{W}_2 / \partial \beta_9 \\
\partial \mathcal{W}_3 / \partial \beta_1 & \cdots & \partial \mathcal{W}_3 / \partial \beta_9
\end{pmatrix}
\end{equation}

would need to be singular, i.e., $\mathrm{rank}(M) < 3$.

However, by Lemma \ref{lem:wardIdentitiesIndependence}, the Ward identities are linearly independent as functionals on the coupling space. Therefore, $\mathrm{rank}(M) = 3$ (at least in a physical subspace).

\textbf{Conclusion:} Ward identity constraints are additional constraints that select a proper subset of fixed points from those satisfying $\beta(g) = 0$. Only the fixed points that satisfy all three Ward identities are physically consistent. This selection mechanism is essential for determining the unique asymptotically safe fixed point in the Standard Model context. \qed

\end{proof}

\end{lemma}
