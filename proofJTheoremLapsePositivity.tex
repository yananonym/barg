% proofThmLapsePositivity.tex
% Proof content

proof_thm_lapse_positivity.tex


\textbf{Lemma (Lapse Function Positivity from Temporal Structure)}

Let $T^\mu$ be the temporal vector field from Theorem \ref{thm:su2WeakStructure}, defined via the gradient of the divergence asymmetry functional:
\[
T^\mu(x) := g^{\mu\nu}(x) \nabla_\nu \mathcal{A}[\psi_0](x),
\]
where $\mathcal{A}[\psi] = D[\psi \| \psi_0] - D[\psi_0 \| \psi]$.

Define the lapse function as the norm of $T$ with respect to the Riemannian metric $g$:
\[
N(x)^2 := g_{\mu\nu}(x) T^\mu(x) T^\nu(x).
\]

\textbf{Claim:} $N(x) > 0$ for $\mu$-almost every $x \in X$.

\textbf{Proof:}

\textit{Step 1: Non-Triviality of Asymmetry Functional.} By Lemma \ref{lem:asymmetryProperties}, the asymmetry functional $\mathcal{A}[\psi_0]$ is non-constant on $X$ (since $V''$ is non-constant by condition V2 and the vacuum $\psi_0$ varies over $X$ by the Polish space structure). Thus the gradient $\nabla \mathcal{A} \neq 0$ on a set of positive measure.

\textit{Step 2: Metric Non-Degeneracy.} By Lemma \ref{lem:uniformMetricNondegeneracy}, the metric $g$ is uniformly non-degenerate with:
\[
\lambda_{\min}(g) \geq \lambda_0 > 0
\]
everywhere on $X$. Therefore, $g_{\mu\nu} T^\mu T^\nu > 0$ whenever $T \neq 0$.

\textit{Step 3: Measure-Theoretic Regularity.} Define the critical set where the temporal vector vanishes:
\[
Z := \{x \in X : T(x) = 0\} = \{x : \nabla \mathcal{A}(x) = 0\}.
\]

The potential $\Phi[\psi_0] = \int_X V(|\psi_0|^2) d\mu$ is strictly convex by condition V2. Thus the asymmetry functional inherits strict convexity properties. By the implicit function theorem applied to $\nabla \mathcal{A} = 0$, combined with non-degeneracy of the Hessian $\nabla^2 \mathcal{A}$ at critical points (following from strict convexity of $\Phi$), the set $Z$ is either empty or has measure zero.

\textit{Step 4: Positivity Conclusion.} On $X \setminus Z$ (which has full measure), there is $T(x) \neq 0$, hence:
\[
N(x)^2 = g_{\mu\nu}(x) T^\mu(x) T^\nu(x) > 0
\]
by Step 2.

Therefore $N(x) > 0$ for $\mu$-almost every $x \in X$. \qed

\textbf{Corollary (Infimum Bound).} On any compact subset $K \subset X$ where $\nabla \mathcal{A}$ is bounded away from zero, there is:
\[
\inf_{x \in K} N(x)^2 = \inf_{x \in K} g_{\mu\nu}(x) T^\mu(x) T^\nu(x) \geq \lambda_0 \cdot c_K^2 > 0,
\]
where $c_K = \min_{x \in K} |\nabla \mathcal{A}(x)|$.

For the full space $X$, the lapse satisfies $N(x) > 0$ everywhere it is defined (i.e., on the spacetime manifold $\mathcal{M} = \Psi_N(X)$), ensuring causality and positive time orientation in the ADM formalism.
